% !TEX root = ../../lectures.tex
\section{Motion of a Charged Particle in a Uniform Magnetic Field}
\label{app:motion-in-B}

We work in c.g.s.\ Gaussian units. A particle of charge $q$ and rest mass $m$ moves with velocity $\vb v$ in a constant magnetic field $\vb B$.
We denote $\beta\equiv v/c$, $\gamma=(1-\beta^2)^{-1/2}$, and $\hat{\vb b}\equiv \vb B/B$ (the unit vector along the field).

The starting point is the Lorentz force,
\[
\dot{\vb p} \equiv \frac{d}{dt}(\gamma m \vb v)
= q\!\left(\vb E + \frac{\vb v}{c}\times \vb B\right).
\]
Throughout this appendix we set $\vb E=0$. This is a good idealization in many astrophysical plasmas: in the fluid rest frame the electric field vanishes because of the extremely high conductivity (ideal MHD). % , and on gyration scales the $\vb v\times \vb B$ term dominates.

A first and very useful observation is that a magnetic field does no work:
\[
\frac{dE}{dt}=\vb v\!\cdot\!\dot{\vb p}=q\,\vb v\!\cdot\!\vb E=0
\quad\Rightarrow\quad
E=\gamma mc^{2}=\text{const}.
\]
Hence, both $\gamma$ and the speed $|\vb v|$ are constant in time. The acceleration is therefore purely geometric: it can only \emph{change the direction} of $\vb v$, not its magnitude.

With $\vb E=0$ the equation $\gamma m\,\vb a=\frac{q}{c}\,\vb v\times\vb B$ immediately gives
\[
\vb v\!\cdot\!\vb a=\frac{q}{\gamma mc}\,\vb v\!\cdot\!(\vb v\times\vb B)=0,
\]
so $\vb a\perp \vb v$: the acceleration is centripetal (curvature) rather than tangential (speed change).

To describe the motion, decompose the velocity into components parallel and perpendicular to $\vb B$,
\[
\vb v = v_\parallel \hat{\vb b} + \vb v_\perp.
\]
Projecting the Lorentz equation along $\vb B$ yields
\[
\frac{d}{dt}(\gamma m v_\parallel)=\frac{q}{c}\,\vb B\cdot(\vb v\times \vb B)=0
\quad\Rightarrow\quad
p_\parallel=\gamma m v_\parallel=\text{const}.
\]
Because $p=\gamma m v$ is also constant (the speed is fixed), the \emph{perpendicular} component is constant as well:
\[
p_\perp=\sqrt{p^2-p_\parallel^2}=p\,\sqrt{1-\mu^2}=\text{const},\qquad
\mu\equiv \frac{p_\parallel}{p}=\cos\theta.
\]
In other words, the \emph{pitch angle} $\theta$ between the momentum and the field line is conserved. 

In the plane orthogonal to $\vb B$ we have
\[
\gamma m\,\frac{d\vb v_\perp}{dt}=\frac{q}{c}\,\vb v_\perp\times \vb B,
\qquad
\frac{dv_\parallel}{dt}=0,
\]
i.e.\ uniform circular motion plus uniform drift along $\vb B$. Choosing coordinates with $\vb B=B\,\hat{\vb z}$ and introducing an initial phase $\phi_0$, an explicit solution is
\[
\begin{aligned}
v_x(t)&=-\,v_\perp \sin(\Omega_B t+\phi_0),&
\quad x(t)&=x_g + r_g \cos(\Omega_B t+\phi_0),\\
v_y(t)&=\phantom{-}\,v_\perp \cos(\Omega_B t+\phi_0),&
\quad y(t)&=y_g + r_g \sin(\Omega_B t+\phi_0),\\
v_z(t)&=v_\parallel=\beta c\,\mu,&
\quad z(t)&=z_g + v_\parallel t,
\end{aligned}
\]
where $(x_g,y_g,z_g)$ are the guiding–center coordinates, and $v_\perp=\Omega_B r_g$ ties the speed to the radius as in any circular motion. The path is a \emph{helix} of fixed radius and constant pitch.

These results introduce two natural scales. The \emph{gyro} (relativistic) angular frequency and the gyroradius are
\begin{remark}
\begin{equation}
\Omega_B=\frac{qB}{\gamma m c}=\frac{\Omega_0}{\gamma},
\qquad
r_g=\frac{v_\perp}{\Omega_B}=\frac{\gamma m c\, v_\perp}{qB}
=\frac{p_\perp c}{qB}~.
\label{eq:omegaB-rg}
\end{equation}
\end{remark}
Here
\[
\Omega_0\equiv \frac{qB}{m c}
\]
is the \emph{cyclotron} (non-relativistic) angular frequency; dividing by $2\pi$ gives ordinary frequencies $\nu_0$ and $\nu_B=\nu_0/\gamma$. The factor of $\gamma$ has a clear interpretation: relativity \emph{slows} the rotation as seen in the lab frame (time dilation), so the frequency drops by $1/\gamma$ while the radius grows by $\gamma$ at fixed $v_\perp$.

Cosmic particles are often relativistic ($v_\perp\simeq c$), for which
\[
r_g\simeq \frac{\gamma m c^2}{qB}\;\approx\;\frac{E}{qB}\,.
\]

\begin{remark}\textbf{Magnetic rigidity.}
%
It is often convenient to define the \emph{magnetic rigidity} $R\equiv pc/|q|$, so that
\[
r_g=\frac{R}{B}~.
\]
For ultra-relativistic particles ($p\simeq E/c$) and charge number $Z=|q|/e$,
\[
r_g \simeq 1.08~\text{pc}\;
\left(\frac{E}{\text{PeV}}\right)
\left(\frac{Z\,B}{\mu\text{G}}\right)^{-1},
\]
a handy back-of-the-envelope rule for Galactic and extragalactic settings.
\end{remark}

For quick reference, Table~\ref{tab:magneticterms} recaps the symbols and their relativistic/non-relativistic forms, and Table~\ref{tab:magneticnumbers} provides order-of-magnitude scales relevant to cosmic-ray physics.

\vspace{0.5cm}

\begin{table}[h!]
\renewcommand{\arraystretch}{1.40}
\begin{center}
\setlength{\tabcolsep}{15pt} 
\begin{tabular}{@{} lll @{}}
\toprule
\textbf{Quantity} & \textbf{Non-relativistic} & \textbf{Relativistic} \\
\midrule
Angular frequency
& $\displaystyle \Omega_0=\frac{qB}{m c}$
& $\displaystyle \Omega_B=\frac{qB}{\gamma m c}=\frac{\Omega_0}{\gamma}$ \\[6pt]
Ordinary frequency
& $\displaystyle \nu_0=\frac{\Omega_0}{2\pi}=\frac{1}{2\pi}\frac{qB}{m c}$
& $\displaystyle \nu_B=\frac{\Omega_B}{2\pi}=\frac{\nu_0}{\gamma}$ \\[6pt]
Gyroradius
& $\displaystyle r_0=\frac{m c\, v_\perp}{qB}$
& $\displaystyle r_g=\frac{\gamma m c\, v_\perp}{qB}=\gamma r_0$ \\
\bottomrule
\end{tabular}
\caption{\textit{Notation used in these notes.} Here $v_\perp$ is the speed perpendicular to $\vb B$; $p_\perp=\gamma m v_\perp$. For $v_\perp\simeq c$, $r_g\simeq \gamma m c^2/(qB)$.}
\label{tab:magneticterms}
\end{center}
\end{table}

\begin{table}[h!]
\renewcommand{\arraystretch}{1.40}
\begin{center}
\setlength{\tabcolsep}{10pt} 
\begin{tabular}{@{} lll @{}}
\toprule
\textbf{Quantity} & \textbf{Galactic CR} & \textbf{UHECR} \\
& $B \sim 1~\mu\mathrm{G}$,\; $E \sim 10~\mathrm{GeV}$ & $B \sim 1~\mathrm{nG}$,\; $E \sim 1~\mathrm{EeV}$ \\
\midrule
Angular freq.\ $\displaystyle \Omega_B$
& $\sim 10^{-3}\ \mathrm{s^{-1}}$
& $\sim 10^{-14}\ \mathrm{s^{-1}}$ \\[4pt]
Ordinary freq.\ $\displaystyle \nu_B$
& $\sim 10^{-4}\ \mathrm{Hz}$
& $\sim 10^{-15}\ \mathrm{Hz}$ \\[4pt]
Gyroradius $\displaystyle r_g$
& $\sim 3\times 10^{13}\ \mathrm{cm}\ \approx\ 10^{-5}\ \mathrm{pc}$
& $\sim 3\times 10^{24}\ \mathrm{cm}\ \approx\ \mathrm{Mpc}$ \\
\bottomrule
\end{tabular}
\caption{\textit{Representative scales} for a proton (\(Z=1\)) with $v_\perp\simeq c$. For general $Z$, $r_g\propto 1/Z$ and $\Omega_B,\nu_B\propto Z$.}
\label{tab:magneticnumbers}
\end{center}
\end{table}

%\paragraph{Covariant derivation of the Liénard power.}
%Let $u^\mu=\gamma(c,\mathbf{v})$ be the four–velocity and 
%\[
%a^\mu \equiv \dv{u^\mu}{\tau}
%\]
%the four–acceleration ($\tau$ proper time). A standard covariant statement of Larmor’s result is that the rate of radiated four–momentum is parallel to $u^\mu$ and proportional to the invariant $a^\nu a_\nu$:
%\begin{equation}
%\dv{P^\mu_{\rm rad}}{\,\tau} \;=\; \frac{2}{3}\,\frac{q^2}{c^3}\,\big(a^\nu a_\nu\big)\,\frac{u^\mu}{c}\,.
%\label{eq:cov-larmor}
%\end{equation}
%Taking the temporal component and converting $d/d\tau$ to $d/dt$ gives
%\begin{equation}
%P \;\equiv\; \dv{E}{t} \;=\; -\,\frac{2}{3}\,\frac{q^2}{c^3}\,a^\nu a_\nu\,,
%\label{eq:P_scalar}
%\end{equation}
%since $u^0=\gamma c$ and $dt=\gamma\,d\tau$. Thus the radiated power in any inertial frame is minus the Lorentz scalar $a^\nu a_\nu$ times $2q^2/c^3$.
%
%To make \eqref{eq:P_scalar} explicit, express $a^\nu a_\nu$ in terms of the lab three–acceleration $\mathbf{a}=d\mathbf{v}/dt$, decomposed into components parallel and perpendicular to $\mathbf{v}$:
%\[
%\mathbf{a}=a_\parallel\,\hat{\mathbf{v}}+ \mathbf{a}_\perp,\qquad 
%a_\parallel\equiv \mathbf{a}\!\cdot\!\hat{\mathbf{v}},\quad 
%\mathbf{a}_\perp\!\cdot\!\hat{\mathbf{v}}=0.
%\]
%A straightforward computation using $u^\mu=(\gamma c,\gamma \mathbf{v})$ yields the invariant
%\begin{equation}
%a^\nu a_\nu \;=\; -\,\gamma^{4}\!\left(a_\perp^{2}+\gamma^{2}a_\parallel^{2}\right).
%\label{eq:ainv_components}
%\end{equation}
%Substituting \eqref{eq:ainv_components} into \eqref{eq:P_scalar} immediately gives the \textbf{Liénard power}:
%\begin{equation}
%\boxed{\;
%P \;=\; \frac{2}{3}\,\frac{q^{2}}{c^{3}}\,\gamma^{4}\!\left(a_\perp^{2}+\gamma^{2}a_\parallel^{2}\right)
%\;=\; \frac{2}{3}\,\frac{q^{2}}{c^{3}}\,\gamma^{6}\!\left(a^{2}-\big(\boldsymbol{\beta}\times\mathbf{a}\big)^{2}\right)\,,
%}
%\label{eq:lienard_power}
%\end{equation}
%where $\boldsymbol{\beta}=\mathbf{v}/c$. This is exactly Liénard’s formula. In the instantaneous rest frame ($\gamma=1$, $a_\parallel=0$, $a_\perp=a'$) one recovers the Larmor result $P'=(2/3)(q^2/c^3)\,a'^2$.

%\paragraph{Invariant way to view it.}
%Introduce the four-velocity \(u^\mu=\gamma(c,\vb v)\) and the four-acceleration \(a^\mu\equiv \dv{u^\mu}{\tau}\) (\(\tau\) proper time). The contraction
%\[
%a^\mu a_\mu \equiv \eta_{\mu\nu} a^\mu a^\nu
%\]
%is a Lorentz scalar (see proof below). In the instantaneous rest frame (IRF) of the particle one has \(a^\mu=(0,\vb a_{\rm IRF})\) and \(a^\mu a_\mu=-a_{\rm IRF}^2\), so \(|a^\mu a_\mu|\) equals the square of the \emph{proper} acceleration. Eq.~\eqref{eq:lienard} can be obtained by evaluating the Larmor power in the IRF and transforming to the lab frame.
%
%\begin{remark}Relativistic invariance of \(a^\mu a_\mu\)
%Let \(a'^\mu=\Lambda^\mu{}_\nu a^\nu\) under any Lorentz transformation \(\Lambda\) with \(\Lambda^T \eta \Lambda=\eta\). Then
%\[
%a'^\mu a'_\mu=(\Lambda a)^T \eta (\Lambda a)=a^T(\Lambda^T\eta \Lambda)a=a^T\eta a=a^\mu a_\mu\, .
%\]
%Hence \(a^\mu a_\mu\) is invariant. (Additionally, \(u^\mu a_\mu=0\) by differentiating \(u^\mu u_\mu=c^2\).)
%\end{remark}
