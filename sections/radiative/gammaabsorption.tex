% !TEX root = ../../lectures.tex
\section{Pair production and gamma-ray absorption}


High-energy $\gamma$-rays emitted by distant sources (AGN, GRBs, star-forming galaxies) do not travel through a vacuum. Along their way, they encounter photons of the diffuse extragalactic radiation fields: the cosmic microwave background (CMB) and the extragalactic background light (EBL) from integrated starlight and dust emission (see Fig.~\ref{fig:ebl}). 

\begin{figure}[!t]
\centering
\includegraphics[width=0.85\textwidth]{photonBackgroundInEnergy.pdf}
\caption{Photon field energy density of the CMB and the extragalactic background light (EBL) as a function of photon energy. 
For the EBL we show three recent models from the literature~\cite{Domingues,Saldana,Gilmore}.}
\label{fig:ebl}
\end{figure}

When a $\gamma$-ray of energy $E_\gamma$ interacts with a background photon of energy $\epsilon$, it can produce an $e^+e^-$ pair,
%
\[
\gamma + \gamma \;\rightarrow\; e^+ + e^-,
\]
%
provided that the center-of-mass (CoM) energy exceeds the rest mass of the pair. This process leads to \emphit{attenuation} of the primary $\gamma$-ray flux, and defines an effective \emphit{$\gamma$-ray horizon}. % as shown in Fig.~\ref{fig:aa21639-13-fig6}.

\paragraph{Threshold condition and target photon energies.}

The pair-production kinematics are controlled by the Lorentz-invariant squared CoM energy,
%
\[
s = (P_a + P_b)^2 = m_a^2 + m_b^2 + 2(E_a E_b - \vb p_a \cdot \vb p_b).
\]
%
For two photons ($m_a = m_b = 0$, $|\vb p| = E/c$) with energies $E_\gamma$ and $\epsilon$ and angle $\theta$ between their momenta,
%
\[
s = 2 E_\gamma \epsilon (1 - \cos\theta),
\]
%
in units where $c=1$. The threshold for producing an $e^+e^-$ pair is when the CoM energy is just enough to create two electrons at rest in the CoM frame,
%
\[ 
s_{\rm th} = (2m_e c^2)^2 = 4 m_e^2 c^4.
\]
%
Thus the threshold condition is
%
\begin{equation}
2 E_\gamma \epsilon (1 - \cos\theta) \;\ge\; 4 m_e^2 c^4.
\label{eq:pair-thr-general}
\end{equation}
For a head-on collision ($\theta = \pi$, $1 - \cos\theta = 2$),
%
\[
4 E_\gamma \epsilon \;\ge\; 4 m_e^2 c^4
\quad\Rightarrow\quad
E_\gamma \epsilon \;\ge\; m_e^2 c^4,
\]
or
%
\begin{tcolorbox}
\begin{equation}
E_\gamma \;\gtrsim\; \frac{(m_e c^2)^2}{\epsilon}.
\label{eq:pair-thr-headon}
\end{equation}
\end{tcolorbox}

As very useful rule-of-thumb, high-energy $\gamma$-rays mainly interact with target photons whose energy satisfies
%
\[
E_\gamma (\text{TeV})\,\epsilon(\text{eV}) \;\gtrsim\; 0.26.
\]

For example:
\begin{itemize}
\item A $1$~TeV photon ($E_\gamma=10^{12}$~eV) interacts at threshold with photons of \(  \epsilon_{\rm th} \simeq 0.26~\text{eV} \), i.e.\ near-infrared photons.
\item A $10$~TeV photon primarily interacts with optical photons ($\epsilon_{\rm th} \sim \text{few eV}$).
\item A $100$~GeV photon interacts most efficiently with UV photons ($\epsilon_{\rm th} \sim \mathcal O (10~\text{eV})$).
\end{itemize}

\paragraph{Center-of-mass velocity and cross-section.}

To describe the pair-production cross-section, it is convenient to introduce the velocity $\beta^*$ of the produced electron (or positron) in the CoM frame. In that frame, the electron and positron have equal and opposite momenta, and each has energy
%
\[
E_e^* = \gamma^* m_e c^2,
\]
%
with
%
\[
\gamma^* = \frac{1}{\sqrt{1-\beta^{*2}}}.
\]
The total CoM energy is
%
\[
\sqrt{s} = 2 E_e^* = 2 \gamma^* m_e c^2
\quad\Rightarrow\quad
\gamma^* = \frac{\sqrt{s}}{2m_e c^2}.
\]
From this,
%
\[
\beta^* = \sqrt{1 - \frac{1}{\gamma^{*2}}}
= \sqrt{1 - \frac{4 m_e^2 c^4}{s}}
= \sqrt{1 - \frac{2 m_e^2 c^4}{E_\gamma \epsilon (1-\cos\theta)}}.
\]

\begin{figure}
\centering
\includegraphics[width=0.65\textwidth]{pair_sigma.pdf}
\label{fig:pairsigma}
\caption{.}
\end{figure}

The exact $\gamma\gamma\to e^+e^-$ cross-section in terms of $\beta^*$ is derived by QED~\TODO{add citation}:
%
\begin{equation}
\sigma_{\gamma \gamma}(\beta^*) 
= \frac{3}{16} \sigma_{\rm T}\, (1-\beta^{*2})
\left[
2 \beta^* (\beta^{*2}-2) 
+ (3-\beta^{*4}) \ln \left( \frac{1+\beta^*}{1-\beta^*} \right)
\right],
\end{equation}
%
where $\sigma_{\rm T}$ is the Thomson cross-section. The threshold corresponds to $\beta^* \to 0$, while the ultra-relativistic limit of the pair corresponds to $\beta^* \to 1$.

A few key features of $\sigma_{\gamma\gamma}$:
\begin{itemize}
\item \emph{Near threshold} ($\beta^* \ll 1$), the cross-section rises steeply from zero.
\item It reaches a maximum value of order
\[ \sigma_{\gamma\gamma}^{\rm max} \simeq 0.25\,\sigma_{\rm T}~, \]
for CoM energies a few times above threshold (\(s \sim \text{few}\times 4m_e^2 c^4\)).
\item In the \emph{high-energy limit} ($\beta^* \to 1$, $s\gg 4m_e^2 c^4$), the cross-section falls off roughly as $1/s$, i.e.\ as the inverse of the product $E_\gamma\epsilon$. A standard asymptotic expression is
\begin{equation}
\sigma_{\gamma\gamma}(\beta^*) \simeq \frac{3}{8} \frac{\sigma_{\rm T}}{\gamma^{*2}}
\left[ \ln(4\gamma^{*2}) - 1\right] 
\propto \frac{1}{\gamma^{*2}} 
\simeq \frac{1}{E_\gamma \epsilon(1-\cos\theta)}.
\end{equation}
\end{itemize}

Physically, this means that $\gamma$ rays can in principle interact with all target photons above the threshold, but the interaction is \emphit{most efficient} when the product $E_\gamma \epsilon$ is only a few times the threshold value. At much higher energies, the cross-section shrinks, and the Universe becomes more transparent again.

\paragraph{Optical depth and the $\gamma$-ray horizon.}

The cumulative effect of $\gamma\gamma$ interactions \emphit{along the line of sight} is encoded in the optical depth $\tau_{\gamma\gamma}(E_\gamma)$, defined as the integral over path length, angles, and target photon energies:
%
\begin{equation}
\tau_{\gamma\gamma}(E_\gamma) 
= \int_0^R \!\!dx 
\int_{4\pi} d\Omega\,(1-\cos\theta)
\int_{\epsilon_{\rm th}}^\infty\! d\epsilon\,
n_\gamma(\epsilon,\Omega,x)\,
\sigma_{\gamma\gamma}(E_\gamma,\epsilon,\cos\theta).
\end{equation}
%
Here:
\begin{itemize}
  \item $x$ is the distance along the line of sight (or, in cosmological applications, a redshift coordinate),
  \item $n_\gamma(\epsilon,\Omega,x)$ is the differential number density of background photons at position $x$ with energy $\epsilon$ and direction $\Omega$,
  \item $(1 - \cos\theta)$ accounts for the angular dependence of the CoM energy,
  \item $\epsilon_{\rm th}$ is the local threshold energy from~\eqref{eq:pair-thr-general}.
\end{itemize}

The observed flux is then attenuated as
%
\[
F_{\rm obs}(E_\gamma) = F_{\rm int}(E_\gamma)\,
\exp\!\big[-\tau_{\gamma\gamma}(E_\gamma)\big],
\]
%
where $F_{\rm int}$ is the intrinsic source spectrum.

For extragalactic sources at redshift $z$, one typically defines the \emph{$\gamma$-ray horizon} $E_{\gamma,\,\rm hor}(z)$ as the energy at which $\tau_{\gamma\gamma}(E_\gamma,z) = 1$. For $E_\gamma \gg E_{\gamma,\,\rm hor}(z)$ the flux is exponentially suppressed. Different EBL models predict different $\tau_{\gamma\gamma}(E_\gamma,z)$ curves; comparisons with AGN spectra yield constraints on the EBL, as illustrated in Fig.~\ref{fig:aa21639-13-fig6}.

\begin{figure}[t]
\centering
\includegraphics[width=0.5\textwidth]{aa21639-13-fig6.pdf}
\caption{$\gamma$-ray horizon for different EBL models. Some lower limits from AGN spectra measurements are shown~\cite{HESS2013aa}.}
\label{fig:aa21639-13-fig6}
\end{figure}

However, an order-of-magnitude $\gamma\gamma$ optical depth can be estimated for a photon from a source at cosmological distance using the simple approximation
\[
\tau_{\gamma\gamma} \;\sim\; \sigma_{\gamma\gamma}\,n_\gamma\,L \, .
\]
Assuming a TeV photon, the head-on threshold condition for pair production implies that the dominant target photons have energies
\[
\epsilon_{\rm th} \simeq 1~{\rm eV},
\]
i.e.\ in the optical/near-IR band. Let us assume that the extragalactic background light (EBL) around $\epsilon\sim 1~{\rm eV}$ has a typical \emph{energy density} (see Fig.~\ref{}):
\[
u_{\rm EBL} \sim 3 \times 10^{-3}~{\rm eV\,cm^{-3}},
\]
comparable to a few percent of the CMB energy density. The corresponding photon number density is then
\[
n_\gamma \sim \frac{u_{\rm EBL}}{\epsilon_{\rm th}} 
\sim 3\times 10^{-3}~{\rm cm^{-3}}.
\]

The pair-production cross-section $\sigma_{\gamma\gamma}(s)$ peaks at a value
\[
\sigma_{\gamma\gamma}^{\rm max} \sim 0.25\,\sigma_{\rm T}
\sim 1.7\times 10^{-25}~{\rm cm^2}.
\]
Requiring $\tau_{\gamma\gamma}\sim 1$ gives a characteristic interaction length
\[
L \sim \frac{1}{\sigma_{\gamma\gamma}\,n_\gamma}
\sim \frac{1}{(1.7\times 10^{-25}~{\rm cm^2})\,(3\times 10^{-3}~{\rm cm^{-3}})}
\sim 2\times 10^{27}~{\rm cm}
\sim 0.6~{\rm Gpc}.
\]

Thus, up to factors of a few, an optical depth $\tau_{\gamma\gamma}\sim 1$ is expected on distance scales of order a gigaparsec. In a flat $\Lambda$CDM cosmology, this is the right ballpark for the proper distance to sources at $z\sim 1$, explaining why TeV $\gamma$ rays from such redshifts are strongly attenuated by pair production on the EBL.

\TODO{add electromagnetic cascades}

\begin{comment}
\subsection{Electromagnetic cascades}

Pair production does not simply \emph{destroy} $\gamma$ rays: it transfers their energy into relativistic $e^\pm$ pairs. These pairs, in turn, radiate via inverse Compton scattering on the same background photon fields (CMB, EBL), generating secondary $\gamma$ rays, which can again undergo pair production if sufficiently energetic. This feedback loop gives rise to \emph{electromagnetic (EM) cascades}.

\paragraph{Basic cascade cycle.}

Consider a primary $\gamma$ ray of energy $E_{\gamma,0}$ injected into an isotropic photon background:
\begin{enumerate}
  \item \textbf{Pair production:} 
  \[
  \gamma(E_{\gamma,0}) + \gamma_{\rm b}(\epsilon) \to e^+ + e^-.
  \]
  Typically, the $e^\pm$ share the energy roughly equally, so each has Lorentz factor
  \[
  \gamma_e \sim \frac{E_{\gamma,0}}{2 m_e c^2}.
  \]
  \item \textbf{Inverse Compton:} Each $e^\pm$ up-scatters background photons via IC. In the Thomson regime (with respect to the background photons), the characteristic energy of the up-scattered photons is
  \[
  E_\gamma^{\rm (IC)} \sim \frac{4}{3}\,\gamma_e^2\,\epsilon_{\rm b},
  \]
  where $\epsilon_{\rm b}$ is a typical background photon energy (e.g.\ CMB or EBL).
  \item \textbf{Further pair production:} If $E_\gamma^{\rm (IC)}$ is still above the pair-production threshold with the background, the newly produced $\gamma$ rays can again convert into $e^+e^-$, and the cycle repeats.
\end{enumerate}

This chain continues until the photon energies fall below the pair-production threshold on the dominant background. At that point, the cascade ``dies out'' and the remaining photons can propagate freely to the observer.

\paragraph{Characteristic energies and universality.}

Consider a primary $\gamma$ ray interacting with the CMB (mean photon energy $\epsilon_{\rm CMB}\sim 6\times 10^{-4}$~eV). For very high primary energies, the first generation of pairs is extremely energetic, and the IC scattering may enter the Klein--Nishina regime. However, after a few generations, the cascade typically settles into the Thomson regime for IC on the CMB, with a characteristic energy where
%
\[
E_\gamma^{\rm (IC)} \lesssim E_{\rm thr}^{(\gamma\gamma)}(\epsilon_{\rm CMB}).
\]
Below this energy, further pair production on the CMB is suppressed, and the cascade saturates. The emergent spectrum tends to a quasi-universal power law (for 1D cascades in a homogeneous background, $dN_\gamma/dE_\gamma \propto E_\gamma^{-1.5}$ is a classic result), largely independent of the detailed shape of the injected spectrum, as long as the injection is sufficiently high in energy.

In more realistic situations (expansion, evolving backgrounds, magnetic fields) the exact spectral shape is modified, but the key qualitative picture remains:
\begin{itemize}
  \item the cascade \emph{reprocesses} very high-energy photons into a broader band of lower-energy photons,
  \item the total energy is conserved (up to adiabatic losses), so the cascade spectrum has a characteristic hardness,
  \item the final $\gamma$ rays form part of the diffuse extragalactic $\gamma$-ray background.
\end{itemize}

\paragraph{Role of magnetic fields.}

Intergalactic magnetic fields influence EM cascades primarily by deflecting the $e^\pm$ pairs:
\begin{itemize}
  \item If the magnetic field is extremely weak, the pairs remain nearly collinear with the original $\gamma$ ray, and the cascade develops in a narrow cone (\emph{1D cascade}). The secondary $\gamma$ rays point back to the source.
  \item If the magnetic field is stronger, the pairs are significantly deflected before they up-scatter photons. The cascade becomes \emph{extended} and \emph{delayed}: the secondary $\gamma$ rays arrive over a broader angular region and a longer timescale. This can wash out the point-like appearance of the source and redistribute its power into a very extended halo.
\end{itemize}
Thus, observations of extended $\gamma$-ray halos and time-delayed emission from flaring sources can, in principle, constrain intergalactic magnetic fields.

\paragraph{Connection to the $\gamma$-ray horizon.}

From the point of view of an observer, pair production and IC cascades together define how the Universe ``filters'' and ``reprocesses'' high-energy photons:
\begin{itemize}
  \item At very high energies, $\gamma$ rays are efficiently absorbed ($\tau_{\gamma\gamma}\gg1$) and their energy is re-emitted at lower energies via cascades.
  \item Near the $\gamma$-ray horizon, one sees a transition where the intrinsic source spectrum and the cascade component can both contribute.
  \item At sufficiently low energies, the Universe is transparent and the spectrum reflects the original source plus any cascade contribution accumulated along the way.
\end{itemize}
In this sense, pair production and EM cascades are the $\gamma$-ray analogue of radiative transfer with absorption and re-emission: they reshape the observed spectrum and angular distribution, but the fundamental physics is again just Compton scattering and pair creation/annihilation in different guises.
\end{comment}
