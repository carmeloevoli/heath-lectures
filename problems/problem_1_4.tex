% !TEX root = ../exercises.tex
\section{Low diffusivity around TeV halos}

TeV halos, extended regions of very high-energy ($E_\gamma \sim$~TeV) gamma-ray emission, have been observed surrounding few middle-aged pulsars, such as Geminga. % ~\cite{2017Sci...358..911A}.

\begin{itemize}
\item Utilizing the known distance to Geminga\footnote{\url{https://en.wikipedia.org/wiki/Geminga}} and given that the angular extension of its TeV halo is approximately \(\theta \sim 5.5^\circ\), calculate the halo's physical size.
\item Assuming electrons are initially emitted from the center of the halo, estimate the local diffusion coefficient, $D$, using the formula $D \sim H^2/\tau$, where $H$ represents the halo size, and $\tau$ is the energy loss timescale. Consider energy losses primarily due to IC scattering on CMB.
\item Compare the result with the Bohm diffusion coefficient in a $\sim 1 \mu$G magnetic field, which is the smaller possible diffusion coefficient.
\item Discuss the scenario in which the gamma-ray emission occurs in the Klein-Nishina regime. Explain the conditions under which the photon field would result in this regime being applicable.
\end{itemize}
