% !TEX root = ../exercises.tex
\section{Cosmic Ray Dynamics in a Starburst Galaxy Nucleus}

\begin{exercise}
Consider a starburst galaxy nucleus approximated as a cylinder of radius \(R = 500\,\mathrm{pc}\) and half‐height \(H = 500\,\mathrm{pc}\), filled with gas of uniform density \(n = 300\,\mathrm{cm}^{-3}\). Supernovae occur at a rate of \(0.1\,\mathrm{yr}^{-1}\), each depositing \(10^{50}\,\mathrm{erg}\) into cosmic‐ray (CR) protons. The proton–proton inelastic cross section is 
\(
\sigma_{pp} = 3 \times 10^{-26}\,\mathrm{cm}^2
\)
(assumed constant with energy). CR diffusion along the \(z\)–axis (perpendicular to the disk) is described by
\(
D(E) \;=\; 3 \times 10^{26} \left(\frac{E}{\mathrm{GeV}}\right)^{1/3} \,\mathrm{cm}^2\,\mathrm{s}^{-1}.
\)

\begin{enumerate}
\item[(a)] Assume that CRs are injected in an infinitely thin disk at \(z = 0\) and solve the transport equation in the \(z\) direction under a free escape boundary condition at \(|z| = H\).
\item[(b)] From your result in part (a), compute the spectrum of CRs escaping at \(|z| = H\) and discuss the two limiting regimes:
\begin{itemize}
\item \emph{Inelastic‐loss dominated:} \(\tau_{\rm loss} \ll \tau_{\rm esc}\).
\item \emph{Escape dominated:} \(\tau_{\rm esc} \ll \tau_{\rm loss}\).
\end{itemize}
\item[(c)] Compare the diffusive escape timescale with the inelastic loss timescale of an Iron nucleus within the same environment, considering its spallation cross-section is \(45 \text{ mb} \times A^{0.7}\). What does this imply about their survival in the starburst nucleus?
\end{enumerate}
\end{exercise}

\begin{solution}
\begin{itemize}
\item[(a)] We wish to solve for the steady–state proton density \(N(E,z)\) (number of protons per unit energy per unit volume) in one dimension (along \(z\)) inside a cylindrical starburst nucleus of half‐height \(H\).  The transport equation reads
\begin{equation}\label{eq:transport21}
- \frac{\partial}{\partial z} \Bigl[\,D(E)\,\frac{\partial N}{\partial z}\Bigr]
\;=\;  
Q(E)\,\delta(z) - \frac{N}{\tau_{\rm in}}
\end{equation}
subject to the boundary conditions
\[
N\bigl(E,\,z = \pm H\bigr) \;=\; 0.
\]
Here \(Q(E)\,\delta(z)\) represents CR injection in an infinitely thin disk at \(z=0\) and \( \tau_{\rm in} = (n\,c\,\sigma_{pp})^{-1} \) is the inelastic loss timescale.

For \(z \neq 0\), the delta‐function vanishes, so the steady–state equation becomes
\[
\frac{\partial^2 N}{\partial z^2} \;=\; \underbrace{\frac{\tau^{-1}_{\rm in}}{D(E)}}_{\displaystyle \alpha^2(E)}\,N 
\, \longrightarrow \, 
\frac{\partial^2N}{\partial z^2} \;=\; \alpha^2\,N.
\]

Over \(0 < z < H\), the homogeneous equation 
\(\tfrac{d^2N}{dz^2} = \alpha^2\,N\)
has the general solution
\[
N(E,z)\;=\; A\,e^{+\alpha\,z} \;+\; B\,e^{-\alpha\,z}, 
\qquad 0<z<H.
\]
We now impose the boundary condition at \(z=H\):
\[
N\bigl(E,\,z=H\bigr) \;=\; A\,e^{\alpha H} \;+\; B\,e^{-\alpha H} \;=\; 0 
\;\;\longrightarrow\;\; 
B \;=\; -\,A\,e^{2\alpha H}.
\]
Hence for \(0 < z < H\),
\[
N(E,z) 
\;=\; A\,\bigl[e^{\alpha z} - e^{2\alpha H}\,e^{-\alpha z}\bigr] 
%\;=\; A\,e^{\alpha (H - z)}\,\bigl[e^{\alpha (\,z - H\,)} - e^{-\alpha (\,z - H\,)}\bigr]
\;=\; A\,\Bigl[e^{\alpha (H - z)} - e^{-\alpha (H - z)}\Bigr].
\]

It is more convenient to rewrite this in terms of hyperbolic functions:
\[
\sinh[x] \;=\; \frac{e^{x} - e^{-x}}{2}, 
\quad\cosh[x] \;=\; \frac{e^{x} + e^{-x}}{2}.
\]
Then one finds
\[
N(E,z) \;=\; 2\,A\,\sinh\bigl[\alpha\,(H - z)\bigr].
\]

We now enforce the matching at \(z=0\).  
%
Since the distribution is even in \(z\), for \(z<0\) we have the same functional form with \(|z|\).  In particular, the value at the midplane is
\[
N\bigl(E,z=0\bigr) 
\;=\; 2\,A\,\sinh\bigl[\alpha\,H\bigr]
\;\equiv\; N_0(E).
\]
Therefore,
\[
A \;=\; \frac{N_0(E)}{2\,\sinh\bigl[\alpha\,H\bigr]}.
\]

Hence, for \(\lvert z\rvert \le H\), the solution holds
\begin{equation}\label{eq:NZ21}
\boxed{N(E,z) \;=\; N_0(E)\, \frac{\sinh\bigl[\alpha(E)\,\bigl(H - \lvert z\rvert\bigr)\bigr]}{\sinh\bigl[\alpha(E)\,H\bigr]}\,.}
\end{equation}

Our remaining task is to determine \(N_0(E)\) by integrating the transport equation~\eqref{eq:transport21} across the delta‐function source at \(z=0\).

Concretely, we integrate equation~\eqref{eq:transport21} over an infinitesimal interval \(\bigl[-\varepsilon,\,+\varepsilon\bigr]\).
%
%all terms that do not involve \(\delta(z)\) remain finite as \(\varepsilon\to 0\), except the first term, which becomes a jump in the diffusive flux.  
%
The first term
\[
\int_{-\varepsilon}^{+\varepsilon} 
\!\!\! -\,\frac{\partial}{\partial z}\Bigl[D\,\partial_{z}N\Bigr] \,dz 
=
-\,\Bigl[D\,\partial_{z}N\Bigr]_{-\varepsilon}^{+\varepsilon}
\;\xrightarrow{\;\varepsilon\to 0\;} 
-\,\Bigl[D\,\partial_{z}N\bigl(E,\,0^{+}\bigr)\Bigr] 
+ \Bigl[D\,\partial_{z}N\bigl(E,\,0^{-}\bigr)\Bigr].
\]
Because \(N(E,z)\) is even in \(z\), we know
\[
\partial_{z}N\bigl(E,\,0^{-}\bigr) 
\;=\; -\,\partial_{z}N\bigl(E,\,0^{+}\bigr).
\]
Hence the jump in the first term becomes
\[
-\,\bigl[D\,\partial_{z}N\bigl(E,\,0^{+}\bigr)\bigr] \;-\; \bigl[-D\,\partial_{z}N\bigl(E,\,0^{+}\bigr)\bigr]
\;=\; -\,2\,D(E)\,\partial_{z}N\bigl(E,\,0^{+}\bigr).
\]
The inelastic‐loss term is continuous over \( z = 0 \) so the integral gives 0, meanwhile,
\[
\int_{-\varepsilon}^{+\varepsilon} Q(E)\,\delta(z)\,dz \;=\; Q(E).
\]

Therefore, matching across \(z=0\) gives
\[
-2\,D(E)\,\left.\frac{\partial N}{\partial z}\right\lvert_{z=0^{+}} 
\;=\; Q(E).
\]

We now compute \(\partial_{z}N\) at \(z \to 0^{+}\).  
%
Differentiating with respect to \(z\):
\[
\frac{\partial N}{\partial z} 
%\;=\; N_0(E)\,\frac{d}{dz}\Bigl[\sinh\bigl(\alpha(H - z)\bigr)\Bigr]\,
%\frac{1}{\sinh(\alpha H)}
\;=\; N_0(E)
\,\Bigl[-\,\alpha\cosh\bigl(\alpha\,[H - z]\bigr)\Bigr]
\,\frac{1}{\sinh(\alpha H)}.
\]
Therefore, at \(z \to 0^{+}\):
\[
\left.\frac{\partial N}{\partial z}\right\lvert_{0^{+}}
\;=\; -\,\frac{\alpha\,N_0(E)\;\cosh\bigl[\alpha\,H\bigr]}
          {\sinh\bigl[\alpha\,H\bigr]}
\;=\; -\,\alpha\,N_0(E)\,\coth\bigl[\alpha\,H\bigr].
\]
Plugging into the jump condition \(2\,D\,(\partial_{z}N|_{0^{+}})=Q(E)\) gives
\begin{equation}\label{eq:N021}
\boxed{N_0(E)\;=\;\frac{Q(E)}{2\,D(E)\,\alpha(E)\,\coth\bigl[\alpha(E)\,H\bigr]}\,.}
\end{equation}

Hence, combining~\eqref{eq:NZ21} and \eqref{eq:N021}, the full solution becomes 
%
\begin{equation}
\boxed{
N(E,z) \;=\; Q(E) \; \frac{H}{2\,D(E)} \; 
\frac{\sinh\bigl[\alpha H \,\bigl(1 - \tfrac{\lvert z\rvert}{H})\bigr]}{\alpha H \coth(\alpha H) \sinh(\alpha H )}\,.
}
\end{equation}

\item[(b)] The (one‐dimensional) diffusive flux of protons at the boundaries \(z = \pm H\) is
\[
\Phi_{\rm esc}(E) \;=\; 
-\,D(E)\,\left.\frac{\partial N}{\partial z}\right\lvert_{z=H^{-}}
\;=\; 
N\bigl(E,0\bigr) \frac{D(E)\,\alpha(E)}{\sinh\bigl[\alpha(E)\,H\bigr]}.
\]

Using~\eqref{eq:N021}, one can derive
\[
\Phi_{\rm esc}(E) \;=\; \frac{Q(E)}{2}\,\mathrm{sech}\!\bigl[\alpha(E)\,H\bigr]~.
\]

Recall the two characteristic timescales:
\[
\tau_{\rm loss}(E) 
\;=\; \frac{1}{n\,c\,\sigma_{pp}}, 
\quad
\tau_{\rm esc}(E) 
\;\sim\; \frac{H^2}{D(E)}
\quad
\longrightarrow 
\quad
\alpha(E)\,H 
\;=\; \sqrt{\frac{\tau_{\rm esc}(E)}{\tau_{\rm loss}(E)}}.
\]

(i) Inelastic‐loss‐dominated regime: \(\tau_{\rm loss} \ll \tau_{\rm esc}\)

In this limit \(\alpha\,H = \sqrt{\tau_{\rm esc}/\tau_{\rm loss}} \gg 1\).  
%
Hence
\[
\alpha H \,\gg 1 
\;\Longrightarrow\; 
\cosh(\alpha H) \approx \tfrac{1}{2}\,e^{\alpha H} \rightarrow \infty~.
\]
%
Therefore
\[
\Phi_{\rm esc}(E) 
\;=\; \frac{Q(E)}{2\,\cosh(\alpha H)}
\;\approx\; Q(E)\,\frac{1}{2}\,e^{-\alpha H}
\;\approx\; \frac{Q(E)}{2}\,e^{-\sqrt{\tau_{\rm esc}/\tau_{\rm loss}}}\;\ll\;Q(E)\,. 
\]
Because \(\alpha H \gg 1\), essentially almost all CR protons lose energy before escaping (i.e.\ they “die” to inelastic collisions), and the escaping flux is exponentially suppressed.

(ii) Escape‐dominated regime: \(\tau_{\rm esc} \ll \tau_{\rm loss}\)

In this opposite limit \(\alpha H = \sqrt{\tau_{\rm esc}/\tau_{\rm loss}} \ll 1\).  
%
Then
\[
\cosh(\alpha H)\approx 1 + \tfrac{1}{2}(\alpha H)^2\approx 1 
\quad(\text{to leading order}).
\]
Hence
\[
\Phi_{\rm esc}(E) \;=\; \frac{Q(E)}{2\,\cosh(\alpha H)} 
\;\approx\; \frac{Q(E)}{2}\,\bigl[1 - \tfrac12\,(\alpha H)^2 + \cdots\bigr]
\;\approx\; \frac{Q(E)}{2}\,.
\]
Physically, almost all injected CR protons escape before undergoing any inelastic collision.  

If \(Q(E)\propto E^{-p}\), then 
\[
\Phi_{\rm esc}(E)\;\propto\; E^{-p}.
\]

Recap:
\begin{itemize}
\item \(\tau_{\rm loss} \ll \tau_{\rm esc}\) (inelastic‐loss dominated):
\begin{equation}
\boxed{\Phi_{\rm esc}(E)\;\approx\;\frac{Q(E)}{2}\,\exp\left[-\,\sqrt{\frac{\tau_{\rm esc}}{\tau_{\rm loss}}} \; \right]}
\end{equation}
\item \(\tau_{\rm esc} \ll \tau_{\rm loss}\) (escape dominated):
\begin{equation}
\boxed{\Phi_{\rm esc}(E)\;\approx\;\frac{Q(E)}{2}}
\end{equation}
\end{itemize}

\end{itemize}
\end{solution}
\end{document}
