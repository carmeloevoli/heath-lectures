% !TEX root = ../..//lectures.tex
\section{Solving the Rankine-Hugoniot Relations}
\label{app:RH}

The Rankine-Hugoniot relations govern the dynamics of shock waves by connecting the physical quantities across the shock front. In this appendix, we solve the Rankine-Hugoniot relations step by step to illustrate the underlying physics and mathematical structure. The relations are:
%
\begin{eqnarray}
\rho_1 u_1 & = & \rho_2 u_2, \label{eq:mass_conservation} \\
\rho_1 u_1^2 + P_1 & = & \rho_2 u_2^2 + P_2, \label{eq:momentum_conservation} \\
\frac{1}{2} u_1^2 + \frac{\gamma}{\gamma - 1} \frac{P_1}{\rho_1} & = & \frac{1}{2} u_2^2 + \frac{\gamma}{\gamma - 1} \frac{P_2}{\rho_2}. \label{eq:energy_conservation}
\end{eqnarray}

Here, \(\rho\) is the mass density, \(u\) the velocity, \(P\) the pressure, and \(\gamma\) the adiabatic index. The subscripts \(1\) and \(2\) denote the upstream and downstream regions, respectively.

To simplify the energy equation \eqref{eq:energy_conservation}, we normalize it by dividing through by \(\frac{1}{2} u_1^2\):
\begin{equation}
1 + \frac{\gamma}{\gamma - 1} \frac{2 P_1}{u_1^2 \rho_1} = \frac{u_2^2}{u_1^2} + \frac{\gamma}{\gamma - 1} \frac{2 P_2}{u_1^2 \rho_2}.
\end{equation}

Using the momentum conservation equation \eqref{eq:momentum_conservation}, we express \(P_2\):
\begin{equation}
P_2 = P_1 + \rho_1 u_1^2 - \rho_2 u_2^2.
\end{equation}

Introducing the Mach number \(\mathcal{M}_1 = u_1 / c_{\text{s}, 1}\), where \(c_{\text{s}, 1}^2 = \gamma P_1 / \rho_1\), we rewrite the normalized energy equation as:
\begin{equation}
1 + \frac{2}{\gamma - 1} \frac{1}{\mathcal{M}_1^2} = \frac{u_2^2}{u_1^2} \left(1 - \frac{2\gamma}{\gamma - 1} \right) + \left( \frac{2}{\gamma - 1} \frac{1}{\mathcal{M}_1^2} + \frac{2\gamma}{\gamma - 1} \right) \frac{\rho_1}{\rho_2}.
\end{equation}

Defining \( x = u_2 / u_1 \), we obtain:
\begin{equation}
x^2 \mathcal{M}_1^2 (\gamma + 1) - 2x (\gamma \mathcal{M}_1^2 + 1) + 2 + (\gamma - 1) \mathcal{M}_1^2 = 0.
\end{equation}
This quadratic equation yields two solutions: \(x = 1\) (trivial) and the physically relevant:
\begin{equation}
\frac{u_2}{u_1} = \frac{(\gamma - 1) \mathcal{M}_1^2 + 2}{(\gamma + 1) \mathcal{M}_1^2}.
\end{equation}

Returning to the energy equation, we express the pressure ratio as:
\begin{equation}
\frac{P_2}{P_1} = \left[1 + \frac{\mathcal{M}_1^2(\gamma - 1)}{2} \right] \frac{u_1}{u_2} - \frac{\mathcal{M}_1^2(\gamma - 1)}{2} \frac{u_2}{u_1}.
\end{equation}
Substituting \(\frac{u_2}{u_1}\) into this expression, we find:
\begin{equation}
\frac{P_2}{P_1} = \frac{2\gamma \mathcal{M}_1^2}{\gamma + 1} - \frac{\gamma - 1}{\gamma + 1}.
\end{equation}

Using the ideal gas law \(P = \rho R T\), we relate pressure, density, and temperature:
\begin{equation}
\frac{T_2}{T_1} = \frac{P_2}{P_1} \frac{\rho_1}{\rho_2} = \frac{P_2}{P_1} \frac{u_2}{u_1}.
\end{equation}
Substituting the previously derived relations:
\begin{equation}
\frac{T_2}{T_1} = \frac{\left[ 2 \gamma \mathcal{M}_1^2 - (\gamma - 1) \right] \left[ (\gamma - 1) \mathcal{M}_1^2 + 2 \right]}{(\gamma + 1)^2 \mathcal{M}_1^2}.
\end{equation}

The Rankine-Hugoniot relations allow us to determine the downstream physical properties \(u_2, P_2, \rho_2, T_2\) as functions of the upstream Mach number \(\mathcal{M}_1\) and the adiabatic index \(\gamma\). 
