% !TEX root = ../../lectures.tex
\section{Basic concepts in plasma astrophysics}

Cosmic particles propagate through, interact with, and can even \emphit{modify} their ambient medium. In most astrophysical environments this medium is not a neutral gas, but an \emphit{ionized magnetized plasma}. 
%
In this section we introduce a few basic plasma concepts that will be used in the rest of these notes.

\paragraph{What is a plasma?}

A \emph{plasma} is often defined as a (quasi-)neutral gas of charged particles that exhibits \emph{collective behaviour}. More explicitly:
%
\begin{itemize}
\item it contains free charges (electrons, ions) over macroscopic scales;
\item on sufficiently large scales it is \emph{quasi-neutral}, i.e.
  \[
  n_e \simeq Z n_i \,,
  \]
  where $n_e$ is the electron density and $n_i$ the ion density with charge $Ze$;
\item its dynamics is controlled by self-consistent electric and magnetic fields generated by the motion of \emph{many} particles, rather than by isolated binary Coulomb collisions.
\end{itemize}

In thermodynamic equilibrium a gas becomes ionized once the thermal energy per particle, $\sim k_{\rm B}T$, is comparable to or larger than the ionization potential of the atoms. 
%
For comparison, the ionization potential of hydrogen (ground state, $n=1$) is $E_{\rm ion} = 13.6$~eV, which corresponds to a temperature of $T \sim 10^5 $~K.

A more precise statement involves the Saha equation~\cite{}, which gives the ionization fraction as a function of temperature and density, but for simplicity we assume that in most astrophysical plasmas of interest (e.g.\ in the ISM or in clusters of galaxies) the degree of ionization is either very close to $1$ or very close to $0$.

The key concept is that the long-range Coulomb force couples each particle to a huge number of neighbours. As a result, perturbations in charge density or current density do not relax via local collisions alone, but can excite large-scale \emphit{collective modes} (waves and instabilities) in the electromagnetic field. For instance, this is what happens when cosmic rays stream through the interstellar plasma and drive the growth of MHD waves, which in turn scatter the cosmic rays.

\paragraph{Examples of plasmas in the Universe.}

Plasmas are ubiquitous in high-energy astrophysics. A few examples:

\begin{itemize}
\item The solar corona and the \emph{solar wind}: a hot plasma flowing out from the Sun, filling the heliosphere.
\item The interstellar medium (ISM) in galaxies: even the ``cold'' phases contain a non-negligible ionized component, and the warm and hot phases are fully ionized.
\item The intracluster and intergalactic medium: rarefied plasmas at temperatures $T \sim 10^7$--$10^8$~K.
\item Accretion disks and relativistic jets around compact objects: strongly magnetized, high-temperature plasmas responsible for non-thermal emission.
\item Supernova remnants and their shocks: collisionless shocks in a magnetized plasma, believed to be main accelerators of Galactic cosmic rays.
\end{itemize}

From the point of view of cosmic-ray physics, the crucial point is that \emphit{most of the baryonic matter that cosmic rays traverse is in a plasma state}. The transport of CRs, their radiative signatures, and the instabilities they can trigger are all controlled by the plasma properties of the ambient medium.

\paragraph{Quasi-neutrality and the Debye length.}

Although a plasma contains free charges, on scales larger than a certain
characteristic length it remains almost electrically neutral.
%
To see how this comes about, consider a small electrostatic perturbation
$\phi$ of an initially neutral, homogeneous plasma, and compute the
resulting response of the electron density $n_e(\phi)$.
%
For simplicity we assume a single ion species with $Z=1$.

The electrostatic potential $\phi$ satisfies Poisson’s equation:
%
\begin{equation}
\nabla^2 \phi = -4\pi \rho = 4\pi e (n_e - n_i)\,.
\end{equation}
%
If the perturbation is weak, electrons can be assumed to follow a Boltzmann distribution in the potential,
%
\begin{equation}
n_e \simeq n_{e0} \exp\left(\frac{e\phi}{k_{\rm B} T_e}\right)
        \simeq n_{e0} \left(1 + \frac{e\phi}{k_{\rm B}T_e}\right)\,,
\end{equation}
%
where $n_{e0}$ is the unperturbed electron density and $T_e$ the electron temperature. Ions, being heavier, often respond more slowly to small-scale, high-frequency perturbations and can be taken as fixed: $n_i \simeq n_{i0} = n_{e0}$.

Inserting this into Poisson’s equation and linearizing in $\phi$ one finds
%
\begin{equation}
\nabla^2 \phi = \frac{\phi}{\lambda_{\rm D}^2}\,,
\end{equation}\label{eq:screenedpoisson}
%
where
%
\begin{tcolorbox}
\begin{equation}
\lambda_{\rm D} \equiv \sqrt{\frac{k_{\rm B} T_e}{4\pi n_{e0} e^2}}
\label{eq:debye_length}
\end{equation}
\end{tcolorbox}
%
is the \emph{Debye length}. 

The Eq.~\eqref{eq:screenedpoisson} is a screened Poisson equation. For a localized perturbation, its solutions decay
exponentially over the scale $\lambda_{\rm D}$. 
%For instance, in one dimension,
%$\partial_x^2 \phi = \phi/\lambda_{\rm D}^2$ has solutions
%$\phi(x) \propto \exp(-|x|/\lambda_{\rm D})$, and in three dimensions the potential
%generated by any compact charge distribution also falls off as
%$\phi(r) \propto \exp(-r/\lambda_{\rm D})$ at large $r$ (see next paragraph).
This means that if, in some region, there is a small local excess of charge, the electrons will rearrange in such a way that the associated electric field is confined within a layer of thickness $\sim \lambda_{\rm D}$ around that region.
%
Beyond this distance, the perturbation is essentially invisible: $|\phi|$ and therefore $|\rho|$ are exponentially small.

In other words, on scales $r \gg \lambda_{\rm D}$ the plasma very efficiently neutralizes any charge imbalance by slightly shifting its electrons. Hence, on
macroscopic scales, the plasma is \emph{quasi-neutral}:
%
\begin{equation}
n_e \simeq n_i \quad \Rightarrow \quad \rho \simeq 0\,.
\end{equation}

\paragraph{Debye shielding.}

A simple illustration of Debye screening is the electrostatic potential around a
test charge $Q$ immersed in a plasma. In vacuum, the potential would be the
usual Coulomb law,
%
\[
\phi_0(r) = \frac{Q}{r}\,.
\]
%
In a plasma, the Poisson equation with the linearized charge response leads
instead to
%
\begin{equation}
\phi(r) = \frac{Q}{r} \exp\left(-\frac{r}{\lambda_{\rm D}}\right)\,.
\end{equation}
%
Thus, the electric field of the charge is exponentially suppressed beyond the
Debye length. This is called \emph{Debye shielding}. The test charge is
surrounded by a ``cloud'' of opposite charges (mostly electrons) which almost
cancels its field outside a sphere of radius $\sim \lambda_{\rm D}$.

For cosmic rays this is important because large-scale \emph{electrostatic}
fields cannot survive in a highly conducting plasma. Any attempt to maintain a
macroscopic static electric field in the plasma rest frame would imply a
macroscopic charge separation; but free charges are mobile and will move along
the field lines, on a very short timescale $\sim\omega_{\rm p}^{-1}$, until the
net field is reduced to the tiny value compatible with quasi-neutrality. In
practice, static electric fields are therefore confined to scales $\lesssim
\lambda_{\rm D}$ that are negligible on the large scales that matter for CR propagation.

As a consequence, cosmic rays in astrophysical plasmas interact mainly with
magnetic fields and with \emph{time-dependent} electromagnetic fields (waves and
turbulence), rather than with persistent, large-scale electrostatic fields. In
ideal MHD language, the dominant electric field is the \emph{induced} field,
$\vec E \simeq -\vec u \times \vec B / c$, associated with the motion of the
magnetized plasma, not a static Coulomb field.

\paragraph{The plasma parameter.}

The Debye length defines a characteristic volume, the \emph{Debye sphere}, of radius $\lambda_{\rm D}$. The number of particles inside this sphere is
%
\begin{equation}
N_{\rm D} \equiv \frac{4\pi}{3} n_e \lambda_{\rm D}^3\,,
\end{equation}
%
and is called the \emph{plasma parameter}. A necessary condition for a gas to behave as a plasma is
%
\begin{equation}
N_{\rm D} \gg 1\, .
\end{equation}
%
Physically, this means that many particles participate in shielding a given charge, so that fluctuations of the charge density are small and collective descriptions (fluid models, kinetic theory with smooth distribution functions) are appropriate.

Equivalently, one can show that $N_{\rm D} \gg 1$ corresponds to the condition that the typical Coulomb potential energy between neighbouring particles is much smaller than their thermal kinetic energy \TODO{prove it}. In this regime the plasma is said to be \emph{weakly coupled}, which is a good approximation for most astrophysical plasmas relevant for cosmic rays.

\paragraph{Plasma oscillations as an example of collective behaviour.}

A very simple example of a collective plasma mode is the \emph{electron plasma oscillation}. Let us consider a cold electron fluid of density $n_0$ moving against a fixed neutralizing ion background. Suppose that electrons are slightly displaced by a distance $\xi(x,t)$ along $x$ from their equilibrium position. This induces a charge density perturbation
%
\begin{equation}
\frac{\delta n_e}{n_0} \simeq -\frac{\partial \xi}{\partial x}\,,
\end{equation}
%
and hence an electric field given by Gauss’ law,
%
\begin{equation}
\frac{\partial E}{\partial x} = 4\pi e \,\delta n_e
= -4\pi e n_0 \frac{\partial \xi}{\partial x}\,.
\end{equation}
%
The equation of motion for the electron fluid element is
%
\begin{equation}
m_e \frac{\partial^2 \xi}{\partial t^2} = -e E\,.
\end{equation}
%
Combining the two equations and assuming spatially uniform perturbations (or, more generally, looking for plane-wave solutions $\propto e^{i(kx-\omega t)}$) leads to
%
\begin{equation}
\frac{\partial^2 \xi}{\partial t^2} + \omega_{\rm p}^2 \,\xi = 0\,,
\end{equation}
%
where
%
\begin{tcolorbox}
\begin{equation}
\omega_{\rm p}^2 \equiv \frac{4\pi n_0 e^2}{m_e}
\label{eq:plasma_frequency}
\end{equation}
\end{tcolorbox}
%
is the \emph{plasma frequency}. The electrons therefore execute coherent oscillations at frequency $\omega_{\rm p}$ in the electrostatic restoring field provided by the charge separation with respect to the ions. 

These oscillations are a purely collective effect: they involve a bulk motion of the electron fluid against the ion background, and the restoring force is provided by the electric field generated by the collective charge displacement, trying to maintain charge neutrality. For cosmic-ray physics, $\omega_{\rm p}$ sets a characteristic high-frequency scale for plasma oscillations and enters many dispersion relations for waves and instabilities in the interstellar and intergalactic medium.

To summarize, the notions of Debye length, quasi-neutrality, plasma parameter and plasma oscillations provide a first quantitative grasp on what distinguishes a plasma from a neutral gas. 
%
In the following sections we will build on these concepts to describe the electromagnetic waves that control the acceleration and transport of cosmic rays.

\paragraph{Collisions and mean free path.}

Real plasmas also experience \emph{collisions}, i.e.\ binary interactions that change
particle velocities. Their importance is controlled by the \emphit{mean free path} and the collision frequency.

%\paragraph{Mean free path and effective Coulomb cross section.}

For neutral particles, collision cross sections are typically of the order of the geometric size of the atoms or molecules, and can be relatively small.
For charged particles, the situation is very different: the Coulomb force is long-range, so particles can significantly deflect each other even when they
pass at relatively large impact parameters. This leads to an \emphit{effective} Coulomb cross section much larger than a geometric nuclear cross section.

The \emphit{mean free path} is defined as
%
\begin{equation}
\lambda_{\rm mfp} = \frac{1}{n\,\sigma}\,,
\end{equation}
%
where $n$ is the number density of targets and $\sigma$ is the relevant
scattering cross section. A plasma is said to be \emph{collisional} on a given
observational scale $L$ if
%
\begin{equation}
\lambda_{\rm mfp} \ll L\,,
\end{equation}
%
and \emph{collisionless} if $\lambda_{\rm mfp} \gg L$.

A simple estimate of the effective Coulomb cross section can be obtained by
requiring that the Coulomb potential energy at some impact parameter $r_c$ be
comparable to the typical kinetic (thermal) energy of an electron,
%
\begin{equation}
\frac{e^2}{r_c} \sim k_{\rm B} T_e \quad \Rightarrow \quad
r_c \sim \frac{e^2}{k_{\rm B} T_e}\,,
\end{equation}
%
This gives an order-of-magnitude cross section
%
\begin{equation}
\sigma_{\rm Coul} \sim \pi r_c^2
\sim \pi \left(\frac{e^2}{k_{\rm B} T_e}\right)^2.
\end{equation}
%
For example, at $T_e \sim 10^6~{\rm K}$ one finds
$\sigma_{\rm Coul} \sim 10^{-18}~{\rm cm}^2$, much larger than typical nuclear
geometric cross sections ($\sim 10^{-26}$–$10^{-24}~{\rm cm}^2$). In reality,
Coulomb scattering is dominated by many small-angle deflections rather than
rare large-angle ones, and a more accurate treatment introduces the
\emph{Coulomb logarithm} (see below), but this estimate already shows that
Coulomb interactions are much more frequent than hard nuclear collisions.

%\paragraph{Collision frequency and Coulomb logarithm.}

The collision frequency $\nu$ for electrons scattering off ions (or other
electrons) can be written schematically as
%
\begin{equation}
\nu \sim n_e\, \sigma_{\rm eff}\, v_e,
\end{equation}
%
where $v_e$ is the typical electron thermal speed and $\sigma_{\rm eff}$ is an
\emph{effective} Coulomb cross section that accounts for the cumulative effect
of many small-angle scatterings. A proper kinetic calculation gives
%
\begin{equation}
\nu \;\propto\; \frac{n_e\, e^4\, \ln\Lambda}{m_e^{1/2} (k_{\rm B} T_e)^{3/2}}\,,
\label{eq:collision_frequency_scaling}
\end{equation}
%
where $\ln\Lambda$ is the \emph{Coulomb logarithm}, typically
%
\begin{equation}
10 \lesssim \ln\Lambda \lesssim 30\,.
\end{equation}
%
The exact numerical coefficient depends on the precise definition of $\nu$ and
on whether electron–ion or electron–electron collisions are considered, but the
scaling $\nu \propto n_e T_e^{-3/2} \ln\Lambda$ is robust.

It is instructive to compare $\nu$ to the plasma frequency $\omega_{\rm p}$.
Since $\omega_{\rm p} \propto n_e^{1/2}$ and $\nu \propto n_e T_e^{-3/2}$,
diffuse, high-temperature plasmas (small $n_e$, large $T_e$) tend to satisfy
%
\begin{equation}
\nu \ll \omega_{\rm p}\,,
\end{equation}
%
so that collisions do not significantly affect plasma oscillations and many
other dynamical processes. Such plasmas are effectively \emph{collisionless}
for the phenomena we are interested in.

\paragraph{Relation to the plasma parameter and Debye length.}

Recall that the Debye length and the plasma parameter are
%
\begin{equation}
\lambda_{\rm D}^2 = \frac{k_{\rm B} T_e}{4\pi n_e e^2}\,,
\qquad
N_{\rm D} = \frac{4\pi}{3} n_e \lambda_{\rm D}^3\,.
\end{equation}
%
One can show (from a more detailed kinetic calculation) that, apart from
logarithmic factors, the mean free path can be expressed in terms of Debye
quantities as
%
\begin{equation}
\lambda_{\rm mfp} \sim \frac{N_{\rm D}}{\ln\Lambda}\,\lambda_{\rm D}\,.
\end{equation}
%
Thus the condition $N_{\rm D} \gg 1$ (a weakly coupled plasma) is equivalent to
having
%
\begin{equation}
\lambda_{\rm mfp} \gg \lambda_{\rm D}\,.
\end{equation}
%
Physically, this means that:
\begin{itemize}
\item electrons move over many Debye lengths between significant Coulomb
      collisions, so they are almost free on the scale over which Debye
      shielding operates;
\item the potential of an ion is not screened by a single electron, but by
      the collective action of many electrons inside the Debye sphere;
\item short-range correlations between individual particles are weak, and the
      plasma can be treated as a smooth continuum with long-range collective
      electromagnetic behaviour.
\end{itemize}

In many astrophysical environments (interstellar medium, intracluster medium,
solar wind, \dots) one indeed has $N_{\rm D} \gg 1$, $\lambda_{\rm mfp} \gg
\lambda_{\rm D}$ and $\nu \ll \omega_{\rm p}$: they are weakly coupled
collisionless plasmas. 
For cosmic rays, however, \emphit{effective} collisions with waves remain crucial, and will play a central role in the theory of cosmic-ray transport and acceleration.

Nice, this is a good place to show that even the WIM is “collective first, collisional later”.

Here’s your revised example with a collisions block added; you can just replace your current snippet with this one:

\paragraph{Order-of-magnitude example: warm ISM as a weakly coupled, nearly collisionless, plasma.}

\TODO{double check}

Consider the warm ionized medium (WIM) of the Galaxy, a typical environment 
traversed by Galactic cosmic rays. A representative choice of parameters is
\[
n_e \sim 0.03~{\rm cm^{-3}}, 
\quad T_e \sim 8000~{\rm K}.
\]

Using Eq.~\eqref{eq:debye_length},
\[
\lambda_{\rm D} 
= \left(\frac{k_{\rm B} T_e}{4\pi n_e e^2}\right)^{1/2}
\sim 3.6\times 10^3~{\rm cm}~\,. % \sim 36~{\rm m}.
\]
On macroscopic astrophysical scales ($\gtrsim$~pc), this is utterly negligible, so the plasma
is extremely well quasi-neutral. 

The number of electrons in a Debye sphere,
\[
N_{\rm D} = \frac{4\pi}{3}n_e \lambda_{\rm D}^3 \sim 6\times 10^9 \gg 1\,,
\]
comfortably satisfies the condition $N_{\rm D} \gg 1$ for a weakly coupled plasma.
This justifies treating the WIM as a smooth fluid or a kinetic medium with a
well-defined distribution function, rather than a collection of a few strongly
interacting charges.

From Eq.~\eqref{eq:plasma_frequency} we get the electron plasma frequency
\[
\omega_{\rm p} = \left(\frac{4\pi n_e e^2}{m_e}\right)^{1/2}
\sim 10^4~{\rm s^{-1}}\,,
\]
corresponding to a frequency
\[
f_{\rm p} = \frac{\omega_{\rm p}}{2\pi} \sim 1.5~{\rm kHz}
\]
and an oscillation period $T_{\rm p} = 2\pi/\omega_{\rm p} \sim 6\times 10^{-4}~{\rm s}$.

\paragraph{Collisions and mean free path.}

Electrons and ions in the WIM also undergo Coulomb collisions. A standard
Spitzer estimate for the electron--ion collision frequency is
\[
\nu_{ei} \simeq 2.9\times 10^{-6}\,
\frac{n_e}{{\rm cm^{-3}}}\,
\frac{\ln\Lambda}{\left(T_e/{\rm eV}\right)^{3/2}}
\;{\rm s^{-1}},
\]
where $\ln\Lambda$ is the Coulomb logarithm (typically $10\text{--}30$) and
$T_e$ is expressed in eV. For the WIM, $T_e \simeq 8000~{\rm K} \simeq 0.7~{\rm eV}$,
$n_e \simeq 0.03~{\rm cm^{-3}}$ and $\ln\Lambda \sim 20$, which give
\[
\nu_{ei} \sim 3\times 10^{-6}~{\rm s^{-1}}\,.
\]
Comparing with the plasma frequency,
\[
\frac{\nu_{ei}}{\omega_{\rm p}} \sim \frac{3\times 10^{-6}}{10^4} \sim 10^{-10}\,,
\]
we see that the plasma reacts to charge separation via collective oscillations
on a timescale $\omega_{\rm p}^{-1}$ that is about ten orders of magnitude
shorter than the collisional timescale $\nu_{ei}^{-1}$.

The electron thermal speed is
\[
v_{{\rm th},e} \sim \left(\frac{2k_{\rm B}T_e}{m_e}\right)^{1/2}
\sim 5\times 10^7~{\rm cm~s^{-1}}\,,
\]
so the corresponding mean free path is
\[
\lambda_{\rm mfp} \sim \frac{v_{{\rm th},e}}{\nu_{ei}}
\sim \frac{5\times 10^7~{\rm cm~s^{-1}}}{3\times 10^{-6}~{\rm s^{-1}}}
\sim 2\times 10^{13}~{\rm cm}
\sim 10^{13}~{\rm cm}\,,
\]
i.e.\ of order one astronomical unit. Comparing with the Debye length,
\[
\frac{\lambda_{\rm mfp}}{\lambda_{\rm D}} \sim \frac{10^{13}~{\rm cm}}{3.6\times 10^3~{\rm cm}}
\sim 3\times 10^9\,,
\]
so an electron travels over \emph{billions} of Debye lengths before suffering a
significant Coulomb deflection.

\medskip
In conclusion, even in a very dilute astrophysical plasma like the WIM, many
particles participate in Debye shielding and the characteristic plasma
timescales are extremely short compared to most dynamical times in the ISM. At
the same time, $\lambda_{\rm mfp} \gg \lambda_{\rm D}$ and $\nu_{ei} \ll
\omega_{\rm p}$, so collisions are negligible for Debye shielding and plasma
oscillations: the WIM behaves as a weakly coupled, nearly collisionless plasma
from the point of view of its collective electromagnetic dynamics. Cosmic rays
therefore propagate in a medium that is both highly conducting and strongly
dominated by collective electromagnetic effects.


\section{Kinetic description: the Vlasov equation}
\label{subsec:vlasov_intro}

So far we have characterized a plasma in terms of macroscopic quantities such as
density and temperature. A more fundamental description is provided by the
\emph{phase-space distribution function}. For each species $s$ (electrons, ions,
cosmic rays, \dots) we define
%
\begin{equation}
f_s(\vec x, \vec v, t)
\end{equation}
%
such that $f_s(\vec x, \vec v, t)\, d^3x\, d^3v$ gives the number of particles of
species $s$ in the phase-space volume element $d^3x\, d^3v$ around $(\vec x,\vec v)$
at time $t$.

Macroscopic quantities are then obtained by taking moments of $f_s$ in velocity
space. In particular, the charge density and current density entering Maxwell’s
equations can be written as
%
\begin{align}
\rho(\vec x,t) &= \sum_s q_s \int d^3v\; f_s(\vec x,\vec v,t)\,,
\\
\vec j(\vec x,t) &= \sum_s q_s \int d^3v\; \vec v\, f_s(\vec x,\vec v,t)\,,
\label{eq:rho_j_from_f}
\end{align}
%
where $q_s$ is the charge of species $s$. Thus, once $f_s$ is known, the sources of
the electromagnetic field are known.

\paragraph{Liouville theorem and the kinetic equation.}

The time evolution of $f_s$ follows from the conservation of particles in phase
space (Liouville’s theorem). In the absence of creation or destruction processes,
the phase-space density is conserved along the trajectories of individual particles:
%
\begin{equation}
\frac{d f_s}{dt} = 0\,.
\end{equation}
%
Writing the total derivative explicitly gives
%
\begin{equation}
\frac{d f_s}{dt}
= \frac{\partial f_s}{\partial t}
+ \frac{d\vec x}{dt} \cdot \vec\nabla_{\!x} f_s
+ \frac{d\vec v}{dt} \cdot \vec\nabla_{\!v} f_s\,.
\label{eq:liouville}
\end{equation}
%
The equations of motion for a particle of species $s$ and mass $m_s$ are
%
\begin{equation}
\frac{d\vec x}{dt} = \vec v\,,
\qquad
\frac{d\vec v}{dt} = \frac{q_s}{m_s}
\left(
\vec E + \frac{\vec v}{c} \times \vec B
\right),
\label{eq:lorentz_eqs}
\end{equation}
%
where $\vec E(\vec x,t)$ and $\vec B(\vec x,t)$ are the local electric and magnetic
fields. Inserting Eq.~\eqref{eq:lorentz_eqs} into Eq.~\eqref{eq:liouville} we obtain
the \emph{kinetic equation}
%
\begin{equation}
\frac{\partial f_s}{\partial t}
+ \vec v \cdot \vec\nabla_{\!x} f_s
+ \frac{q_s}{m_s}
\left(
\vec E + \frac{\vec v}{c} \times \vec B
\right) \cdot \vec\nabla_{\!v} f_s
= C_s[f]\,.
\label{eq:boltzmann_general}
\end{equation}
%
The term $C_s[f]$ on the right-hand side is the \emph{collision operator}, which
accounts for interactions that change the velocity of particles, such as Coulomb
collisions, scattering off waves, or charge-exchange processes. Its explicit form
depends on the physical processes included.

If collisions can be neglected on the timescales and lengthscales of interest,
$C_s[f]$ can be set to zero and Eq.~\eqref{eq:boltzmann_general} reduces to the
\emph{Vlasov equation} (or collisionless Boltzmann equation):
%
\begin{equation}
\frac{\partial f_s}{\partial t}
+ \vec v \cdot \vec\nabla_{\!x} f_s
+ \frac{q_s}{m_s}
\left(
\vec E + \frac{\vec v}{c} \times \vec B
\right) \cdot \vec\nabla_{\!v} f_s
= 0\,.
\label{eq:vlasov_nonrel}
\end{equation}

For cosmic rays the particle velocities are typically relativistic. It is then more
convenient to use the momentum $\vec p = \gamma m_s \vec v$ as phase-space variable
and write $f_s(\vec x,\vec p,t)$. The relativistic Vlasov equation becomes
\begin{tcolorbox}
\begin{equation}
\frac{\partial f_s}{\partial t}
+ \vec v \cdot \vec\nabla_{\!x} f_s
+ q_s \left(
\vec E + \frac{\vec v}{c} \times \vec B
\right) \cdot \vec\nabla_{\!p} f_s
= 0\,,
\label{eq:vlasov_rel}
\end{equation}
\end{tcolorbox}
%
with
%
\begin{equation}
\vec v = \frac{\vec p}{\gamma m_s}
= \frac{c^2 \vec p}{\sqrt{m_s^2 c^2 + p^2}}\,.
\end{equation}
%
In Galactic cosmic-ray transport theory, one often works with reduced distribution
functions obtained by averaging $f_s$ over gyrophase and/or pitch angle, but
Eq.~\eqref{eq:vlasov_rel} remains the starting point.

\paragraph{Coupling to Maxwell equations.}

The electromagnetic fields satisfy Maxwell’s equations with $\rho$ and $\vec j$ given by Eq.~\eqref{eq:rho_j_from_f}. Therefore:

\begin{itemize}
\item For given $\vec E$ and $\vec B$, the Vlasov equation~\eqref{eq:vlasov_rel} determines the evolution of $f_s$ and hence of $\rho$ and $\vec j$:
%
\begin{equation}
\vec E, \vec B
\;\; \underset{\text{(Vlasov)}}{\Longrightarrow} \;\;
\rho(\vec x,t),\; \vec j(\vec x,t),
\end{equation}

\item Conversely, for a given $f_s$ (for all species), Eq.~\eqref{eq:rho_j_from_f} provides the sources for Maxwell’s equations, which then determine $\vec E$ and $\vec B$:
%
\begin{equation}
\vec E, \vec B
\;\; \underset{\text{(Maxwell)}}{\Longleftarrow} \;\;
\rho(\vec x,t),\; \vec j(\vec x,t),
\end{equation}
\end{itemize}

In reality, both must be satisfied \emph{simultaneously}. The system
\emph{Maxwell + Vlasov} is therefore intrinsically \emph{non-linear}: the fields
govern $f_s$, while $f_s$ governs the fields. This non-linearity is at the heart
of many plasma instabilities driven by cosmic rays streaming through the
interstellar medium, where small perturbations in $f_s$ can grow and feed back
onto the fields that scatter the cosmic rays themselves.

Solving the full Maxwell--Vlasov system for all species is a formidable task.
In practice, several approximation strategies are used, depending on the physics
one wants to capture:

\begin{description}
\item[Test-particle approach.]
The electromagnetic fields $\vec E$ and $\vec B$ are assumed to be given
(e.g.\ large-scale Galactic fields or the fields from an MHD simulation).
One solves the kinetic equation for $f_s$ in these prescribed fields, ignoring
the feedback of $f_s$ on $\vec E$ and $\vec B$. This is appropriate when the
species under consideration (e.g.\ high-energy cosmic rays) carries only a tiny
fraction of the total charge and current.

\item[Linear response approach.]
The background distribution $f_s^{(0)}$ is assumed known and stationary, and one
studies small perturbations $\delta f_s$, $\delta \vec E$, $\delta \vec B$.
Linearizing the Maxwell--Vlasov system around the equilibrium allows one to
derive dispersion relations for plasma waves and growth rates for instabilities.
%This is the standard framework for studying cosmic-ray streaming instabilities.

\item[Fluid / MHD approach.]
Often we are not interested in the full velocity dependence of $f_s$, but only
in a few of its moments, such as density, bulk momentum, and energy density.
These are obtained as
\begin{equation}
\{ \rho_s,\, \vec p_s,\, \epsilon_s \}
= \int d^3v\;
\left\{
1,\; \vec v,\; \tfrac{1}{2} v^2
\right\} m_s f_s(\vec x,\vec v,t)\,.
\end{equation}
Taking corresponding moments of the kinetic equation yields continuity,
momentum, and energy equations. With suitable closure relations (equation of
state, approximations on the pressure tensor, \dots) one arrives at a
\emph{magneto-hydrodynamic} (MHD) description of the plasma. \TODO{Add here major references.}
We will adopt this fluid viewpoint in the next section.
\end{description}

\TODO{add relevant sections:} From the perspective of astroparticle physics, we will frequently switch between these levels of description: MHD for the large-scale plasma dynamics, kinetic theory for particle transport and wave excitation, and test-particle approaches for high-energy cosmic rays in given electromagnetic fields.

\section{From kinetic theory to MHD: when is a fluid description valid?}
\label{subsec:kinetic_to_mhd}

The Maxwell--Vlasov system provides, in principle, a complete kinetic
description of a plasma. In practice, for many astrophysical applications we
replace this detailed picture with a much simpler \emph{fluid} or
\emph{magneto-hydrodynamic} (MHD) description. Before introducing the equations
of \emph{ideal MHD}, it is useful to summarize the conditions under which such
a description is justified.

\paragraph{Separation of scales.}

A first requirement is a clear separation between microscopic and macroscopic
scales. For a magnetized plasma, two fundamental microscopic scales are:
%
\begin{itemize}
\item the \emph{Debye length} $\lambda_{\rm D}$, Eq.~\eqref{eq:debye_length}, which controls electrostatic screening,
\item the \emph{Larmor radius} (or gyroradius), Eq.~\eqref{}, of species $s$.
%      \begin{equation}
%      r_{{\rm L},s} = \frac{v_{\perp}}{\Omega_{c,s}} \,,
%      \qquad
%      \Omega_{c,s} = \frac{q_s B}{m_s c}
%      \end{equation}
%      where $v_\perp$ is the velocity perpendicular to $\vec B$ and
%      $\Omega_{c,s}$ is the cyclotron frequency.
\end{itemize}

An MHD description is appropriate when the characteristic macroscopic length
scale $L$ of interest (e.g.\ the size of a supernova remnant shock, a piece of
the interstellar medium) satisfies
%
\begin{equation}
L \gg r_{{\rm L},s},\ \lambda_{\rm D}
\end{equation}
%
for the species that dominate the mass, momentum and charge transport (usually
thermal ions and electrons). In that case, many gyro-orbits and many Debye
lengths fit inside the region of interest, and fine-scale kinetic structure can
be averaged over.

For cosmic rays, the situation is typically the opposite: their Larmor radii
are \emph{comparable} to or larger than the relevant gradients, so they must be
treated kinetically, even though the background plasma can be described by MHD.

\paragraph{Weak coupling and high conductivity.}

As discussed above, a small Coulomb coupling parameter ($N_{\rm D} \gg 1$) implies
that the plasma is \emph{weakly coupled}: collective electromagnetic effects
dominate over strong binary collisions. At the same time, the abundance of free
charges makes the plasma an excellent electrical conductor. On sufficiently
large scales and long times, the electric field in the rest frame of the fluid
tends to be small, and obeys an Ohm-like law,
%
\begin{equation}
\vec E + \frac{\vec u}{c} \times \vec B \simeq \eta\, \vec j + \dots
\end{equation}
%
where $\vec u$ is the bulk fluid velocity and $\eta$ the resistivity (the dots
represent additional terms such as the Hall term, electron-pressure gradients,
etc.). The limit of \emph{ideal MHD} corresponds to taking $\eta \to 0$, so that
%
\begin{equation}
\vec E + \frac{\vec u}{c} \times \vec B \simeq 0\,,
\end{equation}
%
and the magnetic field is ``frozen'' into the plasma flow. We will make this
statement precise in the next section.

The relevance of resistive effects can be quantified by the \emph{magnetic
Reynolds number},
%
\begin{equation}
{\rm R}_m \equiv \frac{U L}{\eta c^2 / 4\pi}\,,
\end{equation}
%
where $U$ and $L$ are typical velocity and length scales. In most astrophysical
plasmas ${\rm R}_m \gg 1$ on large scales, so that ideal MHD is an excellent
approximation, except in thin boundary layers (current sheets, reconnection
regions) where gradients become very large.

\paragraph{Moment hierarchy and closure.}

Starting from the kinetic equation \eqref{eq:vlasov_nonrel} or
\eqref{eq:vlasov_rel}, taking successive moments in velocity space yields an
infinite hierarchy of fluid equations (for density, momentum, energy, pressure
tensor, heat flux, \dots). To obtain a tractable set of equations one must
\emph{close} this hierarchy by prescribing a relation between higher moments and
lower ones (equation of state, isotropic or gyrotropic pressure, adiabatic
indices, etc.).

The simplest closure, assuming an almost isotropic velocity distribution and
neglecting heat fluxes and viscosity, leads to the standard set of \emph{ideal
MHD} equations. These describe the evolution of a single conducting fluid
coupled to the magnetic field. In the following section we briefly derive and
discuss these equations, which will serve as the background description of the
plasma in which cosmic rays propagate.

