% !TEX root = ../exercises.tex
\section{Cosmological horizons}

\begin{itemize}

\item The Intergalactic Medium (IGM) at redshifts $z \sim 6$ is observed to be highly ionized, likely due to radiation from galaxies and quasars. Post-recombination at $z \sim 10^3$, the IGM was almost completely neutral. This observation indicates that reionization of the IGM occurred somewhere $z_r \gtrsim 10$, although the exact timing of this crucial transition remains unknown. 

\begin{itemize}
\item An ionized IGM scatters CMB photons by Thomson scattering. Under the assumption of a uniform Universe with a specified baryon fraction $\Omega_b$ in units of the critical density $\Omega_c$, derive the relation between $\tau_r$ and $z_r$ (\emph{Hint:} Notice that the contribution to $\tau$ is dominated by electrons at high redshifts, so you are allowed to drop $\Omega_\Lambda$).

%and calculate $\tau_r$ assuming a reionization redshift $z_r = 10$ for an Einstein-de Sitter Universe.

\item The inferred $\tau_r$ from observation of the CMB anisotropy by the Planck satellite~\cite{} is $\tau_r = 0.063$. For $H_0 = 70$~km/s/Mpc, $\Omega_{\rm m} = 0.3$, and $\Omega_b = 0.048$, determine $z_r$.

\end{itemize}

\item The Extragalactic Background Light (EBL) is a significant factor in the absorption of $\gamma$-rays from distant astronomical objects, such as blazars, through the mechanism of pair production. 

\begin{itemize}
\item Utilize observations of $\gamma$-rays with energies $E_\gamma \sim$~TeV from a blazar at a given redshift to outline a method to determine a conservative upper limit for the average EBL intensity as a function of $z$. Assume that $dt/dz$ can be approximated by $H_0^{-1}$, and that all EBL photons have the energy where the pair-production cross section is maximized (monochromatic approximation).
\end{itemize}

\end{itemize}
