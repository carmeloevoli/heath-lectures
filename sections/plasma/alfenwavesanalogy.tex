% !TEX root = ../../lectures.tex
%Awesome—your slides are spot-on in spirit: “bend the frozen-in field → magnetic tension pulls back,” exactly like a vibrating string. Below is the detailed, nuts-and-bolts derivation that makes the analogy precise and explains why the Alfvén speed follows the “(tension)/(linear density)” rule (actually (v^2=T/\mu)). I’ll also map each step to what you sketched in the slides. 

\section{Alfven waves in a mechanical analogy}

Start with a uniform plasma at rest with a uniform background field \(\mathbf B_0\). In ideal MHD, the field is \emph{frozen} into the plasma, so if you nudge a fluid element sideways to \(\mathbf B_0\), you bend the field lines. That bend stores magnetic energy, and the field develops a \emph{tension} that tries to straighten the lines, pushing the plasma back — this is the restoring force that launches a wave along \(\mathbf B_0\). 

%# 2) Magnetic tension from Maxwell stress (why there is a “tension”)

The magnetic force density can be written as
\[
\mathbf f = \underbrace{\frac{1}{4\pi}(\mathbf B\cdot\nabla)\mathbf B}_{\text{\tiny magnetic tension}} \underbrace{- \nabla\!\left(\frac{B^2}{8\pi}\right)}_{\text{\tiny gradient of magnetic pressure}}~.
\]
The second term is the \emph{magnetic pressure} (perpendicular squeezing), the first is the \emph{magnetic tension} (pulling \emph{along} field lines). When a field line is curved with radius of curvature (R), the tension produces a sideways restoring force
\[
f_{\perp} \sim \frac{B_0^2}{4\pi} \, \frac{1}{R}.
\]
This is the magnetic analog of a taut string that resists curvature.

%# 3) Reduce to a “string” (flux-tube) model

%Consider a thin flux tube of cross-section (A), mass density (\rho), and let (s) be the coordinate along the unperturbed field ((\hat{\mathbf b}*0\parallel\mathbf B_0)). Give it a small transverse displacement (\xi*\perp(s,t)) (say, in (y)). For small deflections, the curvature is (\kappa \approx \partial^2\xi_\perp/\partial s^2). The **total tension force** on a short segment of length (\Delta s) is then
%\[
%F_\perp ;=; \bigg(\frac{B_0^2}{4\pi}\bigg)A;\frac{\partial^2\xi_\perp}{\partial s^2},\Delta s,
%\]
%because the “tension per unit area along the field” is (B_0^2/4\pi) (cgs), and the geometric factor (\partial^2\xi_\perp/\partial s^2) measures how much the tension vectors at the two ends fail to cancel.
%
%\end{document}
%
%The **mass** of the segment is (\rho,A,\Delta s), so Newton’s 2nd law gives
%[
%\rho A,\Delta s;\frac{\partial^2\xi_\perp}{\partial t^2}
%;=;
%\bigg(\frac{B_0^2}{4\pi}\bigg)A;\frac{\partial^2\xi_\perp}{\partial s^2},\Delta s.
%]
%Cancel (A\Delta s) to obtain the transverse wave equation
%[
%\frac{\partial^2\xi_\perp}{\partial t^2}
%;=;
%\frac{B_0^2}{4\pi \rho},\frac{\partial^2\xi_\perp}{\partial s^2}.
%]

%# 4) Identify (T) and (\mu) → the “string formula”

Compare now with the standard string equation 
\( \partial_t^2 \xi = (T/\mu),\partial_s^2\xi \).
You can read off:
%
\begin{description}
\item[Tension] \(T = (B_0^2/4\pi),A \) — the magnetic tension times the tube area.
\item[Linear mass density] \(\mu = \rho,A\).
\end{description}

Therefore
\[
v_A^2 ;=; \frac{T}{\mu} ;=; \frac{B_0^2}{4\pi\rho}
\qquad\Rightarrow\qquad
\boxed{,v_A=\frac{B_0}{\sqrt{4\pi\rho}},}\quad\text{(cgs)}.
\]
%In SI units the same reasoning gives (v_A=B_0/\sqrt{\mu_0\rho}) because (B^2/4\pi) (cgs) ↔ (B^2/\mu_0) (SI). 

This is exactly the “tension over linear density” rule in your slide (with the square understood: \(c_w^2=T/\mu)\). 

Why propagation is \emph{along} \(\mathbf B_0\) only?

The restoring force comes from \emph{field-line curvature}, so the wave \emph{needs} variation along (s) (the direction of \(\mathbf B_0)\). 
%
If \(k_\parallel=0 \), there’s no curvature, hence no tension restoring force $\rightarrow$ no Alfvén propagation. That yields the dispersion
\[
\omega = \pm k_\parallel v_A,
\]
as you noted on the slide. 

Energy perspective (what’s doing “work”)

Bending the tube increases magnetic energy (magnetic pressure is unchanged to leading order for a pure Alfvén mode; it’s the **tension** piece that matters). During oscillation, magnetic energy and kinetic energy exchange and are equal on average—again matching the lossless string picture you referenced (“kinetic energy” ↔ “field-line bending”). 

Back to your numbers/remarks

%Your slide’s ISM estimate (v_A \simeq B/\sqrt{4\pi\rho_i}) (taking the ion mass density) and the remark that SNR shocks are super-Alfvénic, (v_{\rm SNR}\gg v_A), are exactly the standard scalings. With (B\sim \mathrm{few},\mu\mathrm G) and (n\sim 0.1\text{–}1,\mathrm{cm^{-3}}), one gets (v_A) of order (10\text{–}20,\mathrm{km,s^{-1}}), far below typical SNR shock speeds and of course (\ll c), as you wrote. 

\section{A primer on Alfven waves}

\subsection{Alfv\'en Waves (vs.\ Sound Waves)}

\paragraph{Setting and assumptions.}
We work in (ideal) MHD with a uniform equilibrium:
\[
\mathbf{B}_0=\text{const},\qquad \rho_0=\text{const},\qquad \mathbf{v}_0=\mathbf{0},\qquad p_0=\text{const}.
\]
Small perturbations $\delta \rho,\,\delta p,\,\delta \mathbf{v},\,\delta \mathbf{B}$ vary as $\exp[i(\mathbf{k}\!\cdot\!\mathbf{x}-\omega t)]$.
The ideal MHD equations we linearize are
\[
\begin{aligned}
&\partial_t \rho + \nabla\!\cdot(\rho \mathbf{v}) = 0,\\
&\rho\,\partial_t \mathbf{v} = -\nabla p + \frac{1}{\mu_0}(\nabla\times \mathbf{B})\times \mathbf{B},\\
&\partial_t \mathbf{B} = \nabla\times(\mathbf{v}\times \mathbf{B}),\qquad \nabla\!\cdot\!\mathbf{B}=0,\\
&\delta p = c_s^2\,\delta \rho\quad(\text{adiabatic closure, }c_s^2=\gamma p_0/\rho_0).
\end{aligned}
\]

\paragraph{Alfv\'en waves: definition and dispersion.}
Alfv\'en waves are \emph{transverse, incompressible} MHD waves whose restoring force is the magnetic tension of the background field $\mathbf{B}_0$.
They satisfy
\[
\delta \rho = 0,\qquad \delta p = 0,\qquad \mathbf{k}\cdot \delta\mathbf{v}=0,
\]
with polarization $\delta\mathbf{v}\perp\mathbf{B}_0$ and $\delta\mathbf{B}\perp\mathbf{B}_0$.
Writing $\mathbf{k}=k_\parallel \hat{\mathbf{b}}_0 + \mathbf{k}_\perp$ with $\hat{\mathbf{b}}_0=\mathbf{B}_0/B_0$, the dispersion relation is
\[
\boxed{\ \omega = \pm k_\parallel v_A\ ,\qquad v_A=\frac{B_0}{\sqrt{\mu_0 \rho_0}}\ } \quad
\text{(SI)}\qquad\text{or}\qquad v_A=\frac{B_0}{\sqrt{4\pi \rho_0}}\ \text{(cgs)}.
\]
Key features:
\begin{itemize}
\item Propagate \emph{along field lines}: no propagation if $k_\parallel=0$.
\item \emph{Transverse} polarization: $\delta\mathbf{v}\perp \mathbf{k}$ and $\delta\mathbf{v}\perp \mathbf{B}_0$ (shear Alfv\'en mode).
\item \emph{Incompressible}: no density/pressure perturbations at leading order.
\item Energy equipartition in linear waves: kinetic and magnetic perturbation energies are equal on average.
\item Group and phase velocities are $\pm v_A\,\hat{\mathbf{b}}_0$ (purely along $\mathbf{B}_0$).
\end{itemize}

\paragraph{Physical picture.}
Imagine a taut magnetic ``string'' (field line). A sideways displacement bends it, creating curvature and a \emph{tension} force $\propto B_0^2$ that pulls the plasma back; inertia is provided by $\rho_0$. The wave speed therefore scales as $v_A\propto B_0/\sqrt{\rho_0}$.

\paragraph{Sound waves: contrast.}
Ordinary sound waves in a neutral or weakly magnetized fluid have
\[
\boxed{\ \omega = \pm k\,c_s\ ,\qquad \delta\mathbf{v}\parallel \mathbf{k}\ ,\qquad \text{compressive}~(\delta\rho,\delta p\neq 0)\ }.
\]
They are longitudinal, isotropic (no preferred direction), and the restoring force is the \emph{gas pressure gradient}, not magnetic tension.

\paragraph{Magnetized plasma taxonomy (for context).}
In a magnetized plasma there are three linear MHD modes:
\begin{enumerate}
\item \textbf{Alfv\'en} (transverse, incompressible, $\omega=\pm k_\parallel v_A$).
\item \textbf{Slow magnetosonic} (compressive, generally sub-$\min\{c_s,v_A\}$, strongly field-aligned at low $\beta$).
\item \textbf{Fast magnetosonic} (compressive, more isotropic, super-$\max\{c_s,v_A\}$ at low $\beta$).
\end{enumerate}
The relative importance of these modes depends on the plasma beta
\[
\beta \equiv \frac{2\mu_0 p_0}{B_0^2} \quad \text{(SI)}\qquad \Big(\text{or } \beta=8\pi p_0/B_0^2\ \text{in cgs}\Big).
\]
Low-$\beta$ plasmas ($\beta\ll1$) are tension-dominated ($v_A\gg c_s$); high-$\beta$ plasmas behave more like ordinary fluids ($c_s\gtrsim v_A$).

\paragraph{Polarization and correlations.}
For linear Alfv\'en waves,
\[
\delta\mathbf{B} = \pm \sqrt{\mu_0 \rho_0}\,\delta\mathbf{v}\times \hat{\mathbf{b}}_0,\qquad
\delta\mathbf{E} = -\,\delta\mathbf{v}\times \mathbf{B}_0,
\]
and the Poynting flux $\mathbf{S}=\mu_0^{-1}\,\delta\mathbf{E}\times \delta\mathbf{B}$ points along $\pm\hat{\mathbf{b}}_0$.

\paragraph{Damping (very brief).}
Ideal MHD is dissipationless. In real plasmas, Alfv\'en waves can damp via:
viscosity/resistivity (Ohmic), ion–neutral friction (partially ionized media), phase mixing and resonant absorption (inhomogeneous $v_A$), and at smaller scales via kinetic effects (e.g.\ Landau/cyclotron damping; ``kinetic Alfv\'en'' regime when $k_\perp \rho_i\!\sim\!1$).
%
\paragraph{Worked numbers.}
\begin{itemize}
\item \emph{Warm ISM:} $B_0\!\sim\!5~\mu\text{G}$, $n\!\sim\!1~\text{cm}^{-3}$ $\Rightarrow$ 
$v_A \approx 2.18\,\frac{B_{[\mu\text{G}]}}{\sqrt{n_{[\text{cm}^{-3}]}}}\,\text{km s}^{-1} \approx 11~\text{km s}^{-1}$.
For $T\!\sim\!8000$ K, $c_s\!\approx\!9~\text{km s}^{-1}$, so Alfv\'en and sound speeds are comparable.
\item \emph{Solar corona:} $B_0\!\sim\!10~\text{G}$, $n\!\sim\!10^9~\text{cm}^{-3}$ $\Rightarrow$
$v_A \approx 2.18\,\frac{10^7}{\sqrt{10^9}} \approx 690~\text{km s}^{-1}$, typically $\gg c_s$ ($\beta\ll1$).
\end{itemize}

\paragraph{At-a-glance comparison.}
\begin{center}
\begin{tabular}{lcc}
\toprule
 & \textbf{Alfv\'en wave} & \textbf{Sound wave} \\
\midrule
Restoring force & Magnetic tension ($\propto B_0^2$) & Gas pressure gradient \\
Compressibility & Incompressible ($\delta \rho=0$) & Compressive ($\delta \rho\neq 0$) \\
Polarization & Transverse ($\delta\mathbf{v}\perp \mathbf{k},\mathbf{B}_0$) & Longitudinal ($\delta\mathbf{v}\parallel \mathbf{k}$) \\
Dispersion & $\omega=\pm k_\parallel v_A$ & $\omega=\pm k\,c_s$ \\
Anisotropy & Propagates along $\mathbf{B}_0$ & Isotropic \\
Energy flux & Along field lines & Along $\mathbf{k}$ \\
Key parameter & $v_A=B_0/\sqrt{\mu_0\rho_0}$ & $c_s=\sqrt{\gamma p_0/\rho_0}$ \\
\bottomrule
\end{tabular}
\end{center}

\paragraph{Derivation sketch.}
Taking $\mathbf{k}$ in the $x$–$z$ plane with $\mathbf{B}_0=B_0\hat{\mathbf{z}}$ and seeking a transverse solution with $\delta v_y\neq 0$, the linearized momentum and induction equations give
\[
-i\omega \rho_0\,\delta v_y = \frac{i k_\parallel B_0}{\mu_0}\,\delta B_y,\qquad
-i\omega\,\delta B_y = i k_\parallel B_0\,\delta v_y,
\]
which combine to yield $\omega^2 = k_\parallel^2 v_A^2$ and the phase relation 
$\delta B_y = \pm \sqrt{\mu_0\rho_0}\,\delta v_y$.

\paragraph{Where they matter.}
Alfv\'en waves are ubiquitous in magnetized astrophysical plasmas: solar wind/corona, planetary magnetospheres, interstellar turbulence, and accretion/ejection environments. They mediate energy and momentum transport along field lines and are central to plasma heating, turbulence cascades, and cosmic-ray scattering.

\paragraph{Exercises.}
\begin{enumerate}
\item Show that the time-averaged kinetic and magnetic perturbation energies of a linear Alfv\'en wave are equal.
\item For a plasma with $\beta=0.1$ and $c_s=100~\text{km s}^{-1}$, estimate $v_A$ and discuss which linear mode transports energy most efficiently across vs.\ along the mean field.
\item Starting from the linearized MHD system, derive the full magnetosonic dispersion relation for arbitrary angle between $\mathbf{k}$ and $\mathbf{B}_0$, and identify the slow/fast branches in the limits $\beta\ll1$ and $\beta\gg1$.
\end{enumerate}


\end{document}

%---
%
%## One-line takeaway
%
%Treat a thin flux tube as a string: magnetic tension (T\sim B_0^2 A/4\pi) pulls it taut; inertia is the tube’s linear density (\mu=\rho A); the transverse wave speed is (v_A=\sqrt{T/\mu}=B_0/\sqrt{4\pi\rho}) (or (B_0/\sqrt{\mu_0\rho}) in SI). That’s why your “tension / linear density” rule works for Alfvén waves. 
%
%If you want, I can turn this into a small annotated figure (string vs. flux-tube) to drop straight into the slides.
