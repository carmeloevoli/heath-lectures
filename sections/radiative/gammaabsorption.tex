% !TEX root = ../../lectures.tex
\section{Pair production and gamma-ray absorption}

\begin{figure}[t]
\centering
\includegraphics[width=0.5\textwidth]{figures/aa21639-13-fig6.pdf}
\caption{$\gamma$-ray horizon for different EBL models. Some lower limits from AGN spectra measurements are shown~\cite{HESS2013aa}.}
\end{figure}

High-energy $\gamma$ rays emitted by distant sources (AGN, GRBs, star-forming galaxies) do not travel through a vacuum. Along their way, they encounter photons of the diffuse extragalactic radiation fields: the cosmic microwave background (CMB) and the extragalactic background light (EBL) from integrated starlight and dust emission. 

{\color{red}Add a plot with photon backgrounds in eV cm-3}

When a $\gamma$ ray of energy $E_\gamma$ interacts with a background photon of energy $\epsilon$, it can produce an $e^+e^-$ pair,
%
\[
\gamma + \gamma \;\rightarrow\; e^+ + e^-,
\]
%
provided that the center-of-mass (CoM) energy exceeds the rest mass of the pair. This process leads to \emph{attenuation} of the primary $\gamma$-ray flux, and defines an effective ``$\gamma$-ray horizon'' as shown in Fig.~\ref{fig:aa21639-13-fig6}.

\paragraph{Threshold condition and target photon energies.}

The pair-production kinematics are controlled by the Lorentz-invariant squared CoM energy,
%
\[
s = (P_a + P_b)^2 = m_a^2 + m_b^2 + 2(E_a E_b - \vb p_a \cdot \vb p_b).
\]
%
For two photons ($m_a = m_b = 0$, $|\vb p| = E/c$) with energies $E_\gamma$ and $\epsilon$ and angle $\theta$ between their momenta,
%
\[
s = 2 E_\gamma \epsilon (1 - \cos\theta),
\]
%
in units where $c=1$. The threshold for producing an $e^+e^-$ pair is when the CoM energy is just enough to create two electrons at rest in the CoM frame,
%
\[
s_{\rm th} = (2m_e c^2)^2 = 4 m_e^2 c^4.
\]
%
Thus the threshold condition is
%
\begin{equation}
2 E_\gamma \epsilon (1 - \cos\theta) \;\ge\; 4 m_e^2 c^4.
\label{eq:pair-thr-general}
\end{equation}
For a head-on collision ($\theta = \pi$, $1 - \cos\theta = 2$),
%
\[
4 E_\gamma \epsilon \;\ge\; 4 m_e^2 c^4
\quad\Rightarrow\quad
E_\gamma \epsilon \;\ge\; m_e^2 c^4,
\]
or
%
\begin{tcolorbox}
\begin{equation}
E_\gamma \;\gtrsim\; \frac{(m_e c^2)^2}{\epsilon}.
\label{eq:pair-thr-headon}
\end{equation}
\end{tcolorbox}

As very useful rule-of-thumb, high-energy $\gamma$-rays mainly interact with target photons whose energy satisfies
%
\[
E_\gamma (\text{TeV})\,\epsilon(\text{eV}) \;\gtrsim\; 0.26.
\]

For example:
\TODO{TO BE CHECKED}
\begin{itemize}
\item A $1$~TeV photon ($E_\gamma=10^{12}$~eV) interacts at threshold with photons of \(  \epsilon_{\rm th} \simeq 0.26~\text{eV} \), i.e.\ near-infrared photons.
\item A $10$~TeV photon primarily interacts with optical photons ($\epsilon_{\rm th} \sim \text{few eV}$).
\item A $100$~GeV photon interacts most efficiently with UV photons ($\epsilon_{\rm th} \sim \mathcal O (10~\text{eV})$).
\end{itemize}

\paragraph{Center-of-mass velocity and cross-section.}

To describe the pair-production cross-section, it is convenient to introduce the velocity $\beta^*$ of the produced electron (or positron) in the CoM frame. In that frame, the electron and positron have equal and opposite momenta, and each has energy
%
\[
E_e^* = \gamma^* m_e c^2,
\]
%
with
%
\[
\gamma^* = \frac{1}{\sqrt{1-\beta^{*2}}}.
\]
The total CoM energy is
%
\[
\sqrt{s} = 2 E_e^* = 2 \gamma^* m_e c^2
\quad\Rightarrow\quad
\gamma^* = \frac{\sqrt{s}}{2m_e c^2}.
\]
From this,
%
\[
\beta^* = \sqrt{1 - \frac{1}{\gamma^{*2}}}
= \sqrt{1 - \frac{4 m_e^2 c^4}{s}}
= \sqrt{1 - \frac{2 m_e^2 c^4}{E_\gamma \epsilon (1-\cos\theta)}}.
\]

The exact $\gamma\gamma\to e^+e^-$ cross-section in terms of $\beta^*$ is derived by QED~\TODO{add citation}:
%
\begin{equation}
\sigma_{\gamma \gamma}(\beta^*) 
= \frac{3}{16} \sigma_{\rm T}\, (1-\beta^{*2})
\left[
2 \beta^* (\beta^{*2}-2) 
+ (3-\beta^{*4}) \ln \left( \frac{1+\beta^*}{1-\beta^*} \right)
\right],
\end{equation}
%
where $\sigma_{\rm T}$ is the Thomson cross-section. The threshold corresponds to $\beta^* \to 0$, while the ultra-relativistic limit of the pair corresponds to $\beta^* \to 1$.

A few key features of $\sigma_{\gamma\gamma}$:
\begin{itemize}
\item \emph{Near threshold} ($\beta^* \ll 1$), the cross-section rises steeply from zero.
\item It reaches a maximum value of order
\[ \sigma_{\gamma\gamma}^{\rm max} \simeq 0.2\text{--}0.3\,\sigma_{\rm T}~, \]
for CoM energies a few times above threshold (\(s \sim \text{few}\times 4m_e^2 c^4\)).
\item In the \emph{high-energy limit} ($\beta^* \to 1$, $s\gg 4m_e^2 c^4$), the cross-section falls off roughly as $1/s$, i.e.\ as the inverse of the product $E_\gamma\epsilon$. A standard asymptotic expression is
\begin{equation}
\sigma_{\gamma\gamma}(\beta^*) \simeq \frac{3}{8} \frac{\sigma_{\rm T}}{\gamma^{*2}}
\left[ \ln(4\gamma^{*2}) - 1\right] 
\propto \frac{1}{\gamma^{*2}} 
\simeq \frac{1}{E_\gamma \epsilon(1-\cos\theta)}.
\end{equation}
\end{itemize}

Physically, this means that $\gamma$ rays can in principle interact with all target photons above the threshold, but the interaction is \emphit{most efficient} when the product $E_\gamma \epsilon$ is only a few times the threshold value. At much higher energies, the cross-section shrinks, and the Universe becomes more transparent again.

\TODO{do this} This property combined with that one...
A particularly useful rule of thumb: a $1$~TeV photon interacts \emphit{most efficiently} with target photons of energy $\epsilon \sim 1$~eV (optical/near-IR), not with much higher-energy photons, even though they are well above threshold. This “near-threshold preference” is what ties the TeV $\gamma$-ray horizon so directly to the EBL spectrum. This is why TeV $\gamma$-rays are good probes of the infrared/optical EBL: the EBL photons provide exactly the right energies to trigger pair production and evenutally absorb primary photons on cosmological scales.

\paragraph{Optical depth and the $\gamma$-ray horizon.}

The cumulative effect of $\gamma\gamma$ interactions \emphit{along the line of sight} is encoded in the optical depth $\tau_{\gamma\gamma}(E_\gamma)$, defined as the integral over path length, angles, and target photon energies:
%
\begin{equation}
\tau_{\gamma\gamma}(E_\gamma) 
= \int_0^R \!\!dx 
\int_{4\pi} d\Omega\,(1-\cos\theta)
\int_{\epsilon_{\rm th}}^\infty\! d\epsilon\,
n_\gamma(\epsilon,\Omega,x)\,
\sigma_{\gamma\gamma}(E_\gamma,\epsilon,\cos\theta).
\end{equation}
%
Here:
\begin{itemize}
  \item $x$ is the distance along the line of sight (or, in cosmological applications, a redshift coordinate),
  \item $n_\gamma(\epsilon,\Omega,x)$ is the differential number density of background photons at position $x$ with energy $\epsilon$ and direction $\Omega$,
  \item \TODO{capire meglio, chi e theta}$(1 - \cos\theta)$ accounts for the angular dependence of the CoM energy,
  \item $\epsilon_{\rm th}$ is the local threshold energy from Eq.~\eqref{eq:pair-thr-general}.
\end{itemize}

The observed flux is then attenuated as
%
\[
F_{\rm obs}(E_\gamma) = F_{\rm int}(E_\gamma)\,
\exp\!\big[-\tau_{\gamma\gamma}(E_\gamma)\big],
\]
%
where $F_{\rm int}$ is the intrinsic source spectrum.

However, an order-of-magnitude $\gamma\gamma$ optical depth can be estimated photon from a source at cosmological distance using the simple approximation
\[
\tau_{\gamma\gamma} \;\sim\; \sigma_{\gamma\gamma}\,n_\gamma\,L \, .
\]
Assuming a TeV photon, Using the head-on threshold condition for $\gamma\gamma\to e^+e^-$, at the dominant target photons have energies
\[
\epsilon_{\rm th} \simeq 0.26~{\rm eV}\,,
\]
i.e.\ in the optical/near-IR band. Assume that the extragalactic background light (EBL) around $\epsilon\sim 1~{\rm eV}$ has a typical \emph{energy density}
\[
u_{\rm EBL} \sim 10^{-3}~{\rm eV\,cm^{-3}}
\]
(comparable to a few percent of the CMB energy density).

The exact pair-production cross-section $\sigma_{\gamma\gamma}(s)$ peaks at a value
\[
\sigma_{\gamma\gamma}^{\rm max} \sim 0.1\,\sigma_{\rm T}.
\]
%For an order-of-magnitude estimate, take
%\[
%\sigma_{\gamma\gamma} \;\sim\; 0.1\,\sigma_{\rm T},
%\qquad
%\sigma_{\rm T} = 6.65\times 10^{-25}~{\rm cm^2}.
%\]

Therefore an optical depth \tau \sim 1 is obtained for sources at a distance L \sim ... which in a  flat $\Lambda$CDM cosmology, the proper distance to $z\sim 1$ 

For extragalactic sources at redshift $z$, one typically defines the \emph{$\gamma$-ray horizon} $E_{\gamma,\,\rm hor}(z)$ as the energy at which $\tau_{\gamma\gamma}(E_\gamma,z) = 1$. For $E_\gamma \gg E_{\gamma,\,\rm hor}(z)$ the flux is exponentially suppressed. Different EBL models predict different $\tau_{\gamma\gamma}(E_\gamma,z)$ curves; comparisons with AGN spectra yield constraints on the EBL, as illustrated in Fig.~\ref{fig:aa21639-13-fig6}.

%\begin{exercise}[

%Order-of-magnitude $\gamma\gamma$ optical depth from a source at $z\sim 1$

\end{document}

An Order-of-magnitude $\gamma\gamma$ optical depth can be estimates the $\gamma\gamma$ optical depth for a TeV photon from a source at cosmological distance using the simple approximation
\[
\tau_{\gamma\gamma} \;\sim\; \sigma_{\gamma\gamma}\,n_\gamma\,L \, .
\]

\paragraph{(a) Threshold target photons.}
Consider a $\gamma$ ray of observed energy $E_\gamma = 1~{\rm TeV}$.
Using the head-on threshold condition for $\gamma\gamma\to e^+e^-$,
\[
E_\gamma\,\epsilon_{\rm th} \simeq (m_e c^2)^2 \, ,
\]
show that the dominant target photons have energies
\[
\epsilon_{\rm th} \simeq 0.26~{\rm eV}\,,
\]
i.e.\ in the optical/near-IR band.  (Numerically evaluate using
$m_e c^2 = 0.511~{\rm MeV}$.)

\paragraph{(b) Number density of target photons.}
Assume that the extragalactic background light (EBL) around $\epsilon\sim 1~{\rm eV}$ has a typical \emph{energy density}
\[
u_{\rm EBL} \sim 10^{-3}~{\rm eV\,cm^{-3}}
\]
(comparable to a few percent of the CMB energy density).
Using $u_{\rm EBL} \simeq n_\gamma\,\epsilon$ with $\epsilon\sim 1~{\rm eV}$, estimate the corresponding photon number density $n_\gamma$ of the target EBL photons.

\paragraph{(c) Cross-section near threshold.}
The exact pair-production cross-section $\sigma_{\gamma\gamma}(s)$ peaks at a value
\[
\sigma_{\gamma\gamma}^{\rm max} \sim 0.1\text{--}0.3\,\sigma_{\rm T}.
\]
For an order-of-magnitude estimate, take
\[
\sigma_{\gamma\gamma} \;\sim\; 0.1\,\sigma_{\rm T},
\qquad
\sigma_{\rm T} = 6.65\times 10^{-25}~{\rm cm^2}.
\]
Compute $\sigma_{\gamma\gamma}$ numerically in cm$^2$.

\paragraph{(d) Distance to a $z\sim 1$ source.}
For a flat $\Lambda$CDM cosmology, the proper distance to $z\sim 1$ is of order a few Gpc. For an order-of-magnitude estimate, take
\[
L \;\sim\; 3~{\rm Gpc}
\qquad\Rightarrow\qquad
L \sim 10^{28}~{\rm cm}.
\]
(You may check that $1~{\rm Gpc} \simeq 3\times 10^{27}~{\rm cm}$.)

\paragraph{(e) Optical depth and transparency.}
Using
\[
\tau_{\gamma\gamma}(E_\gamma) \sim \sigma_{\gamma\gamma}\,n_\gamma\,L,
\]
combine your estimates from parts (b)--(d) and compute $\tau_{\gamma\gamma}$ for a $1$~TeV photon from a source at $z\sim 1$.

\begin{itemize}
  \item Is $\tau_{\gamma\gamma}\ll 1$, $\sim 1$, or $\gg 1$?
  \item Would you expect the Universe to be transparent or opaque to such photons?
\end{itemize}

\paragraph{(f) Sensitivity to assumptions.}
Briefly discuss how your estimate would change if:
\begin{itemize}
  \item the EBL energy density were larger/smaller by a factor of $3$,
  \item the effective cross-section were closer to $0.2\,\sigma_{\rm T}$ instead of $0.1\,\sigma_{\rm T}$.
\end{itemize}
Would these changes affect your qualitative conclusion (transparent vs opaque)?


Optical depth for $\gamma\gamma$ absorption from a source at $z\simeq 1$

In this exercise you will estimate the optical depth for $\gamma\gamma$ absorption,
$\tau_{\gamma\gamma}(E_0,z_0)$, for gamma rays emitted by a source at cosmological
redshift $z_0\simeq 1$ and observed with energy $E_0$.

\paragraph{1. General expression.}
Consider a flat $\Lambda$CDM cosmology with
\[
H_0 = 70~{\rm km\,s^{-1}\,Mpc^{-1}},\qquad
\Omega_m = 0.3,\qquad
\Omega_\Lambda = 0.7.
\]
(a) Show that the $\gamma\gamma$ optical depth for a photon observed with energy $E_0$
from a source at redshift $z_0$ can be written as
\[
\tau_{\gamma\gamma}(E_0,z_0)
= \int_0^{z_0} dz\,
\frac{dl}{dz} \int_{-1}^{1} d\mu\,\frac{1-\mu}{2}
\int_{\epsilon_{\rm th}}^\infty d\epsilon\,
n_\gamma(\epsilon,z)\,
\sigma_{\gamma\gamma}\!\big(s(E(z),\epsilon,\mu)\big),
\]
where
\[
\frac{dl}{dz} = \frac{c}{H_0}\,
\frac{1}{(1+z)\sqrt{\Omega_m(1+z)^3+\Omega_\Lambda}}
\]
is the line element, $E(z)=E_0(1+z)$ is the gamma-ray energy at redshift $z$,
$\mu=\cos\theta$ is the angle between the $\gamma$ ray and the target photon, and
$s=2E\epsilon(1-\mu)$ is the squared CoM energy (in units with $c=1$). Derive the
threshold condition
\[
\epsilon_{\rm th}(E,\mu,z)
= \frac{(m_e c^2)^2}{E(z)(1-\mu)},
\]
starting from the condition $s\ge 4m_e^2c^4$.

\paragraph{2. Toy EBL model and simplifying assumptions.}
To make progress, adopt the following \emph{highly simplified} model for the
extragalactic background light (EBL):

\begin{itemize}
\item[(i)] The target photon spectrum is monochromatic in energy and isotropic in angle:
\[
n_\gamma(\epsilon,z) = n_0\,(1+z)^3\,\delta(\epsilon-\epsilon_0),
\]
with present-day number density $n_0 = 10^{-2}~{\rm cm^{-3}}$ at $\epsilon_0=1~{\rm eV}$.
\item[(ii)] The cross-section is approximated as a ``box'':
\[
\sigma_{\gamma\gamma}(s) \simeq
\begin{cases}
\sigma_{\rm max} \simeq 0.2\,\sigma_{\rm T}, & s \ge 4 m_e^2 c^4, \\
0, & s < 4 m_e^2 c^4.
\end{cases}
\]
\end{itemize}

(b) Using this toy model, show that the $\epsilon$-integral reduces to a condition on
the angle $\mu$:
\[
\epsilon_0 \ge \epsilon_{\rm th}(E(z),\mu,z)
\quad\Longleftrightarrow\quad
1-\mu \ge \frac{(m_e c^2)^2}{E(z)\,\epsilon_0}.
\]
Interpret this condition in terms of which collision angles contribute to the
absorption at a given $E(z)$.

(c) For fixed $z$ and $E(z)$, define $\mu_{\max}(z)$ as the largest value of $\mu$
that still allows pair production:
\[
\mu_{\max}(z) = 1 - \frac{(m_e c^2)^2}{E(z)\,\epsilon_0}.
\]
Show that the angular integral reduces to
\[
\int_{-1}^1 d\mu\,\frac{1-\mu}{2}\,
\Theta\!\big(\mu_{\max}(z)-\mu\big)
= \frac{1}{2}
\int_{-1}^{\mu_{\max}(z)} (1-\mu)\,d\mu,
\]
and perform the integral explicitly in terms of $\mu_{\max}(z)$.

\paragraph{3. Redshift dependence and energy scaling.}

(d) Combining parts (a)--(c), show that the optical depth in this toy model can be
written as
\[
\tau_{\gamma\gamma}(E_0,z_0)
\simeq \sigma_{\rm max}\,n_0\,\frac{c}{H_0}\,
\int_0^{z_0} dz\,
\frac{(1+z)^2}{\sqrt{\Omega_m(1+z)^3+\Omega_\Lambda}}\,
\mathcal{A}\!\left[\frac{E_0(1+z)\,\epsilon_0}{(m_e c^2)^2}\right],
\]
where $\mathcal{A}(y)$ is a dimensionless angular factor you should derive explicitly
from part (c), and $y$ encodes how far above threshold the interaction is.

(e) Evaluate $\tau_{\gamma\gamma}(E_0,z_0)$ numerically (or by a rough analytical
estimate) for:
\begin{itemize}
  \item $z_0 = 1$ and $E_0 = 100~{\rm GeV}$,
  \item $z_0 = 1$ and $E_0 = 1~{\rm TeV}$.
\end{itemize}
In each case, first check whether the head-on threshold condition
$E_0(1+z)\,\epsilon_0 \gtrsim (m_e c^2)^2$ is satisfied over a significant fraction of
the redshift range. (You may approximate the redshift integral by evaluating the
integrand at a representative $z\simeq z_0/2$ and multiplying by an effective path
length.)

(f) Compare your estimates to the heuristic statement that the Universe becomes
optically thick ($\tau_{\gamma\gamma}\gtrsim 1$) to TeV photons from sources at
$z\sim 1$. Does your toy model produce optical depths of the right order of
magnitude? Discuss at least two limitations of the model (e.g.\ monocromatic EBL,
neglect of EBL evolution, box-shaped cross-section) and how each might change
$\tau_{\gamma\gamma}(E_0,z_0)$ in a more realistic calculation.

\subsection{Electromagnetic cascades}

Pair production does not simply \emph{destroy} $\gamma$ rays: it transfers their energy into relativistic $e^\pm$ pairs. These pairs, in turn, radiate via inverse Compton scattering on the same background photon fields (CMB, EBL), generating secondary $\gamma$ rays, which can again undergo pair production if sufficiently energetic. This feedback loop gives rise to \emph{electromagnetic (EM) cascades}.

\paragraph{Basic cascade cycle.}

Consider a primary $\gamma$ ray of energy $E_{\gamma,0}$ injected into an isotropic photon background:
\begin{enumerate}
  \item \textbf{Pair production:} 
  \[
  \gamma(E_{\gamma,0}) + \gamma_{\rm b}(\epsilon) \to e^+ + e^-.
  \]
  Typically, the $e^\pm$ share the energy roughly equally, so each has Lorentz factor
  \[
  \gamma_e \sim \frac{E_{\gamma,0}}{2 m_e c^2}.
  \]
  \item \textbf{Inverse Compton:} Each $e^\pm$ up-scatters background photons via IC. In the Thomson regime (with respect to the background photons), the characteristic energy of the up-scattered photons is
  \[
  E_\gamma^{\rm (IC)} \sim \frac{4}{3}\,\gamma_e^2\,\epsilon_{\rm b},
  \]
  where $\epsilon_{\rm b}$ is a typical background photon energy (e.g.\ CMB or EBL).
  \item \textbf{Further pair production:} If $E_\gamma^{\rm (IC)}$ is still above the pair-production threshold with the background, the newly produced $\gamma$ rays can again convert into $e^+e^-$, and the cycle repeats.
\end{enumerate}

This chain continues until the photon energies fall below the pair-production threshold on the dominant background. At that point, the cascade ``dies out'' and the remaining photons can propagate freely to the observer.

\paragraph{Characteristic energies and universality.}

Consider a primary $\gamma$ ray interacting with the CMB (mean photon energy $\epsilon_{\rm CMB}\sim 6\times 10^{-4}$~eV). For very high primary energies, the first generation of pairs is extremely energetic, and the IC scattering may enter the Klein--Nishina regime. However, after a few generations, the cascade typically settles into the Thomson regime for IC on the CMB, with a characteristic energy where
%
\[
E_\gamma^{\rm (IC)} \lesssim E_{\rm thr}^{(\gamma\gamma)}(\epsilon_{\rm CMB}).
\]
Below this energy, further pair production on the CMB is suppressed, and the cascade saturates. The emergent spectrum tends to a quasi-universal power law (for 1D cascades in a homogeneous background, $dN_\gamma/dE_\gamma \propto E_\gamma^{-1.5}$ is a classic result), largely independent of the detailed shape of the injected spectrum, as long as the injection is sufficiently high in energy.

In more realistic situations (expansion, evolving backgrounds, magnetic fields) the exact spectral shape is modified, but the key qualitative picture remains:
\begin{itemize}
  \item the cascade \emph{reprocesses} very high-energy photons into a broader band of lower-energy photons,
  \item the total energy is conserved (up to adiabatic losses), so the cascade spectrum has a characteristic hardness,
  \item the final $\gamma$ rays form part of the diffuse extragalactic $\gamma$-ray background.
\end{itemize}

\paragraph{Role of magnetic fields.}

Intergalactic magnetic fields influence EM cascades primarily by deflecting the $e^\pm$ pairs:
\begin{itemize}
  \item If the magnetic field is extremely weak, the pairs remain nearly collinear with the original $\gamma$ ray, and the cascade develops in a narrow cone (\emph{1D cascade}). The secondary $\gamma$ rays point back to the source.
  \item If the magnetic field is stronger, the pairs are significantly deflected before they up-scatter photons. The cascade becomes \emph{extended} and \emph{delayed}: the secondary $\gamma$ rays arrive over a broader angular region and a longer timescale. This can wash out the point-like appearance of the source and redistribute its power into a very extended halo.
\end{itemize}
Thus, observations of extended $\gamma$-ray halos and time-delayed emission from flaring sources can, in principle, constrain intergalactic magnetic fields.

\paragraph{Connection to the $\gamma$-ray horizon.}

From the point of view of an observer, pair production and IC cascades together define how the Universe ``filters'' and ``reprocesses'' high-energy photons:
\begin{itemize}
  \item At very high energies, $\gamma$ rays are efficiently absorbed ($\tau_{\gamma\gamma}\gg1$) and their energy is re-emitted at lower energies via cascades.
  \item Near the $\gamma$-ray horizon, one sees a transition where the intrinsic source spectrum and the cascade component can both contribute.
  \item At sufficiently low energies, the Universe is transparent and the spectrum reflects the original source plus any cascade contribution accumulated along the way.
\end{itemize}
In this sense, pair production and EM cascades are the $\gamma$-ray analogue of radiative transfer with absorption and re-emission: they reshape the observed spectrum and angular distribution, but the fundamental physics is again just Compton scattering and pair creation/annihilation in different guises.

\end{document}

\section{Pair production and Gamma-ray Absorption}

\begin{figure}[t]
\centering
\includegraphics[width=0.5\textwidth]{figures/aa21639-13-fig6.pdf}
\caption{$\gamma$-ray horizon for differen EBL models. Some lower limits from AGN spectra measurements are shown~\cite{HESS2013aa}.}
\end{figure}

Extra-galactic gamma-rays undergo absorption during intergalactic propagation by interacting with photons in the diffuse radiation field, producing electron-positron pairs (\(\gamma + \gamma \rightarrow e^+ + e^-\)). This process depends on the energy threshold condition for opacity.

The square of the center-of-mass (COM) energy, \( s \), is a relativistic invariant:
%
\[
s = (P_a + P_b)^2 = m_a^2 + m_b^2 + 2(E_a E_b - \vb p_a \cdot \vb p_b) = m_a^2 + m_b^2 + 2E_a E_b (1 - \beta_a \beta_b \cos \theta)    
\]

For head-on collisions in the pair-production process \(\gamma + \gamma \rightarrow e^+ e^-\), the threshold energy in the LAB frame is:
%
\[
s = 2 E_\gamma \epsilon (1 + 1) = (2 m_e)^2 \rightarrow 4E_\gamma \epsilon = (2 m_e)^2 \rightarrow E_\gamma > \frac{m_e^2}{\epsilon}    
\]

For instance, a 1 TeV photon (\(E_\gamma = 10^{12}\) eV) interacts at the threshold with infrared photons (\(\epsilon \gtrsim 0.26\) eV).

The cross-section for pair-production, \(\sigma_{\gamma \gamma}(\beta^*)\), is given by:
%
\[
\sigma_{\gamma \gamma}(\beta^*) = \frac{3}{16} \sigma_{\rm T} (1-\beta^{*2}) \left[2 \beta^* (\beta^{*2}-2) + (3-\beta^{*4}) \ln \left( \frac{1+\beta^*}{1-\beta^*} \right) \right]
\]
%
where \(\beta^*\) is the velocity of the electron (or positron) in the CoM frame. 

The velocity \(\beta^*\) is determined by comparing the CoM energy with the energy in the CoM frame:
%
\[
2E_\gamma \epsilon(1- \cos\theta) = 4 E_e^{*2} \rightarrow \beta^* = \sqrt{1 - \frac{2 m_e^2 c^4}{E_\gamma \epsilon (1-\cos\theta)}}
\]
%
where I used
%
\[
E_e^* = \gamma^* m_e c^2 = \frac{1}{(1-\beta^{*2})^{1/2}} m_e c^2 = \sqrt{1 - \frac{2 m_e c^4}{x}}
\]

The cross-section reaches a maximum at \(x = 4 m_e c^4\), corresponding to \(\sigma(x) \simeq \sigma_{\rm T}/4\).

A 1 TeV photon most efficiently interacts with \(\sim 1\) eV photons. In the high-energy limit, the cross-section becomes inversely proportional to the energy product:
%
\[
\sigma_{\gamma\gamma}(\beta^*) \simeq \frac{3}{8} \frac{\sigma_{\rm T}}{\gamma^{*2}} \left[ \ln(4\gamma^{*2}) - 1\right] \propto \frac{1}{\gamma^{*2}} \simeq \frac{1}{E_\gamma \epsilon}
\]

That means that $\gamma$-rays can interact with all photons above the threshold but the cross-section decreases as $\epsilon$ increases (near threshold process).

The optical depth for \(\gamma\gamma\) absorption, \(\tau_{\gamma\gamma}(E_\gamma)\), takes into account all photons above the threshold:
%
\[
\tau_{\gamma\gamma}(E_\gamma) = \int_0^R \int_{4\pi} d\Omega (1-\cos\theta) \int_{\epsilon_{\rm th}}^\infty d\epsilon n_\gamma(\epsilon, \Omega, x) \sigma_{\gamma\gamma}(E_\gamma, \epsilon, \cos\theta)
\]

\subsection{Electromagnetic cascades}

TO BE DONE

%%% PLOT WITH BACKGROUND FIELDS

%%% PLOT WITH SIGMA(X)

%%% PLOT WITH OPTICAL DEPTH